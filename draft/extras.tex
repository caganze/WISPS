/( \citealt{1997ApJ...483..103S,Bovy2012},{\citealt{Cheng2012} These experiments have unveiled structures in the Galaxy including the delineation of tidal streams with red giants (\citealt{Hayden2015}), sites of star formation mapped by young accreting stars; the segregation of the thin disk, thick disk, bulge and halo components of the Milky Way through the kinematics and chemical properties of normal main sequence stars. 
(Irwin & Totten 1998)

 Their long lifetimes and ubiquity make them potential tracers for broader Galactic structure, but their intrinsic faintness (L $\lesssim$ $10^{-3}$ L$_\sun$ ) make them of limited use beyond the immediate Solar Neighborhood (distance $\lesssim$100 pc). 
 
 More recently, \cite{Manjavacas2018}  used \wat and \meth  indices in this wavelength region to identify objects with composite atmospheres in a targeted sample of UCDs observed with the WFC3 instrument. \cite{2014ApJ...794..143B} had previously used other near-infrared( NIR, $\lambda$= 0.8-2.5 $\micron$)  spectral indices to  select UCDs binary populations while \cite{2013ApJ...772...79A} have used spectral indices to characterize low-gravity UCDs.  Thus, spectral indices are a useful yet simple method of selecting populations/sub-types of UCDs. 