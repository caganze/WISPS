\documentclass[manuscript]{aastex}
\usepackage{natbib}
\usepackage{graphics}
\usepackage{pdfpages}
\makeatletter
\AtBeginDocument{\let\LS@rot\@undefined}
\usepackage{amstext} % for \text macro
\usepackage{array}   % for \newcolumntype macro
\newcolumntype{L}{>{$}l<{$}} % math-mode version of "l" column type
\makeatother
\begin{document}
\newcommand{\meth}{CH$_4 $ }
\newcommand{\wat}{H$_2$O }
\newcommand{\teff}{T$_{eff}$ }
\newcommand{\Msun}{M$_\sun$}
\newcommand{\figfolder}{/Users/caganze/research/wisps/figures/}

\title{Ultra-cool Dwarfs in Deep HST/WFC3 Parallel Fields. I. XXX M, L and T Dwarfs from WISPS and HST-3D}
\author{Christian Aganze}
\affil{University of California, San Diego, 9500 Gilman Dr, La Jolla, CA 92093 USA \email{caganze@ucsd.edu} }
\author{Adam J. Burgasser }
\affil{University of California, San Diego, 9500 Gilman Dr, La Jolla, CA 92093 USA }
\author{Matt Malkan}
\affil{University of California, Los Angles, COMPLETE}
\author{OTHERS?}
%{}

\begin{abstract}
\end{abstract}

\section{Background}

%why star count experiments are relevant
The structure and evolution of our Milky Way is probed by specific tracer populations of stars,
which are numerous, distinct, and present in regions of interest. Examples include the delineation of tidal streams with red giants (REFS); the identification of young associations and moving groups with low-mass X-ray stars (REFS); and the segregation of the thin disk, thick disk, bulge and halo components of the Milky Way through the kinematics and chemical properties of normal main sequence stars \citep[NEED MORE]{1997ApJ...483..103S,Bovy2012}. [PROBABLY SOMETHING HERE TO THE DEGREE THAT WIDE FIELD SURVEYS WITH SPECTROSCOPY HAVE FACILITATED THIS - GCS, SDSS]
%Questions relating to the structure and evolution of the Milky Way remain relevant in various fields of modern Astronomical research. At the same time, major upgrades in technology---with the launch of the Hubble Space Telescope and several other ground-based surveys--- have created an era of robust empirical characterization of galactic stellar populations. Star counts experiments (\citealt{1985A&A...145....1B}, \citealt{1993ApJ...409..635R}, \citealt{1995AJ....110.1838R}, \citealt{1996MNRAS.283..666M}, \citealt{2001ApJ...553..184C}, \citealt{2007AJ....134.2418B}, \citealt{2008ApJ...673..864J}) that probe the solar neighborhood and beyond reveal that the milky way stellar populations can be classified into the thin disk, thick disk and halo components ().

Ultracool dwarfs (UCDs; M $\lesssim$ 0.1~\Msun, {\teff} $\lesssim$ 3000~K;  \citealt{2005ARA&A..43..195K}) are a potentially useful.
The lowest-mass stars and brown dwarfs, members of the late-M, L, T and Y spectral classes, 
constitute $\sim$50\% of the total number of stars in the galaxy (CHABRIER2003), and have distinct spectra shaped by strong molecular absorption that are highly sensitive to temperature, surface gravity and metallicity. In addition, the cooling evolution of substellar UCDs provides potential age diagnosti9 cs that have been exploited in stellar cluster studies \citep{1998ASPC..134..394B,luhman2012,martin2017} and searches of young moving groups near the Sun (REFS). The long lifetimes and ubiquity of UCDs make them potential tracers for broader Galactic studies, but the intrinsic faintness of these sources has limited their use beyond the immediate Solar Neighborhood.

{\bf ADAM STOPPED HERE}

%but not in older galactic population such as the thick disk and halo. This is because the majority of known halo UCDs are metal-poor (\citealt{1997PASP..109..849G}, \citealt{2007ApJ...657..494B}), old and fast-moving with different evolutionary track compared to those identified in the disk (\citealt{2013ApJS..205....6M}, \citealt{2014MNRAS.437.1009P}, \citealt{0004-637X-787-2-126}). There exist few constraints on their characteristics given that only a few of the nearest halo ultracool dwarfs have been found and characterized. Thus, probing large distances is necessary step in order to study properties of these halo objects.
 
 %previous studies
 %Previous UCD wide-field red optical and infrared surveys (DENIS: \citealt{refId0}, 2MASS: \citealt{2007AJ....133..439C}, \citealt{2010ApJS..190..100K}, WISE: \citealt{2011ApJS..197...19K},  UKIDSS: \citealt{Marocco01062015}, \citealt{2013MNRAS.430.1171D}, \citealt{2013MNRAS.433..457B}, CFBDS: \citealt{2010A&A...522A.112R}) have measured the local density of UCDs (1--10$\times$10$^{-3}$ pc$^{-3}$ for spectral types L0--T9; Bardalez Gagliuffi et al. in prep, \citealt{Marocco01062015}) in a shallow ($\sim$25 pc) volume. These measurements put constraints on the birthrate, mass function and luminosity function (\citealt{2001AJ....121.1710R}, \citealt{2004ApJ...614L..73B}, \citealt{2003ApJ...586L.133C}, \citealt{2001MNRAS.321..699K}, \citealt{2010AJ....139.2679B}) but they are distance-limited. 

Deep pencil-beam surveys can help find distant populations UCDs  including thick disk and halo objects. These surveys  generally target extra galactic sources but are  contaminated by UCDs (\citealt{2008A&A...488..181C}), hence studies  (\citealt{Konishi2014}, \citealt{stanway16}, \citealt{Vledder2016}, \citealt{Ryan2011} ) have attempted to identify them using imaging or spectroscopic techniques.

%imaging surveys
Imaging surveys probe UCDs at larger distances, but with less fidelity in classification. \cite{Ryan2011} found 17 late M, L and T dwarfs in 231.90 arcmin$^2$ of  WFC3 imaging of the GOODS fields using a combination of wide and narrow-band filter colors. In addition to poor estimate of spectral types, the sample was contaminated with various non-stellar sources that could not be identified in the absence of spectral information. They estimated a disk scale height of 290$\pm$39pc consistent with previous studies (\citealt{2005ApJ...631L.159R}, \citealt{2005ApJ...622..319P}). \cite{Holwerda2014} identified  274 in 227 arcmin$^2$ M-dwarfs (to a limiting magnitude F125W=25) from the HST-WFC3 Brightest of Re-ionizing Galaxies (BoRG, \citealt{2009ApJ...695.1591P}) survey, using an optical and near-infrared colors and determined their spectral types  V-J color-M-dwarf subtype relation (\citealt{2009ApJ...695.1591P}). They found a slightly higher density of M-dwarfs identified in the Northern fields compared to the Southern Fields, and a  disk scale-height of 0.3--4kpc with a dependence on subtype. The overall M-dwarf scale height was $\sim$600 pc, a number that is much larger than previous estimates mostly due to large uncertainties in the fit. \cite{Vledder2016} reanalyzed these data using a Markov Chain Monte Carlo method to fit the statistic to a galactic model including a thin disk, thick disk, and halo component. They derived a scale height of $290^{+20}_{-19}$ pc at and a  central number density of  $0.29^{+0.20}_{-0.13}$ pc$^{-3}$, with no correlation of model parameters with M-dwarf subtype. This study has provided the best measurement of galactic disk and halo parameters for M-dwarfs, but does not probe statistics for later types. Spectral types for sources in imaging surveys measure the scale height and radial distribution as they correlate with distances via spectral-type color relations. In addition, UCDs can cool to planetary-like temperatures (100K-600K) in $\sim$10 Gyr. Indeed the coldest, oldest and lowest mass UCDs, the Y dwarfs, have only recently been found, and only at distances very close to the sun (\citealt{2011ApJ...743...50C}, \citealt{2017ApJ...842..118L}). Ultimately, large uncertainties on spectral types  of UCDs in imaging surveys poorly constrain their distances, and deep spectroscopic follow-up of these sources is not a priority for precious HST time.

%spectroscopic surveys
 Major advances in this area has been the move to near-infrared (NIR) deep spectra over wide areas of the sky. NIR spectroscopy samples the peak of UCD SEDs of broad molecular features that anchor classification schemes (e.g \citealt{2005ARA&A..43..195K}, \citealt{2006ApJ...637.1067B}). \citet{2005ApJ...622..319P} identified 18 M and 2L dwarfs in the Hubble Ultra Deep Field (HUDF) and estimated their spectral types by fitting templates from \citet{Kirkpatrick2000} to their Gradient-Assisted Photon Echo Spectroscopy (GRAPES) spectra in the optical wavelength regime. This study estimated a disk scale height of 400 $\pm$ 100 pc for M and L dwarfs. Another study by \citet{2009ApJ...695.1591P} used deep ACS slitless grism observations of the Probing Evolution And Reionization Spectroscopically (PEARS) fields (a part the Great Observatories Origins Deep Survey (GOODS) fields, \citealt{Giavalisco2004}) down to a z=25 and spectroscopically identified 43 M4-M9 dwarfs. Using a thick and thin model, the study estimated a scale height for the thin disk of  $\sim$370 pc, and $\sim$100pc  for the thick disk, a halo fraction between 0.00025--0.0005 consistent with previous estimates. 

%masters et al. 2012
\citealt{2012ApJ...752L..14M} serendipitously discovered 3 late T-dwarfs in the WFC3 infrared Spectroscopic Survey ( WISPS) fields (\citealt{2010ApJ...723..104A}) identified by their strong \meth and \wat absorption features. The sample size was not large enough to put meaningful constraints on the luminosity function of UCDs. In this paper, we expand on this study by developing an effective method to select UCDs in similar surveys.

%sections in this paper
Section 2 describes the observations and the data, section 3 describes the selection process, results are discussed in section 4 and 5.


\section{Observations \& Data}

%description of the WFC3 camera
We obtain data from two WFC3 parallel surveys: the WFC3 Infrared Spectroscopic Parallel Survey WISPS (\citealt{2010ApJ...723..104A}) and 3D-HST ( \citealt{Momcheva2016}, \citealt{2012ApJS..200...13B}, \citealt{Skelton2014}). These two surveys used the IR channel of the WFC3 camera (\citealt{doi:10.1117/12.789581}). The WFC3 IR channel provides a plate scale of 0$\farcs$13 pixel$^{-1}$, over a total field of view of 123 $\farcs$ $\times$ 136$\farcs$; and low-resolution G102 ($\lambda$ = 0.8--1.17 $\micron$, R$\sim$210) and G141  ($\lambda$ = 1.11--1.67  $\micron$, R $\sim$130) grism spectra re extracted from these data. Direct imaging in both surveys is conducted using broadband filters F110W (matching the G141 coverage), F140W (matching the G141 coverage) and F160W. 
%description of both surveys and fields
The WISP survey is a 1000-orbit HST parallel survey covering 390 fields ($\sim$1500 arcmin$^2$) that follows observing programs accepted on the Cosmic Origins Spectrograph (COS) and Space Telescope Imaging Spectrograph (STIS). All WISP survey fields are imaged in F110W and F140W filters with integration times spanning 5000-2000s. The 3D-HST consists of  248-orbits with 24 pointings, spanning $\sim$600 arcmin$^2$ as part of Hubble Cycles 18 \& 19. This survey targets four extragalactic fields: The All-wavelength Extended Groth Strip International Survey (AEGIS, \citealt{1538-4357-660-1-L1} ), Cosmic Evolution Survey (COSMOS, ), Ultra-Deep Survey(UKIDSS-UDS, \citealt{2007MNRAS.379.1599L}), the Great Observatories Origins Deep Survey (GOODS-South and GOODS-North, \citealt{Giavalisco2004}), using the ACS/G800L and WFC3/G141 grisms in parallel. All 3D-HST fields are imaged using filters F140W down to $26$ mag  and/or F160W down to $25$ mag. We report a detailed list observations for all the fields in Table \ref{tbl:obs} as well as a sky map of all the pointings. In addition to WFC3 imaging, the 3D-HST survey also provides ACS F814W imaging data and G800L ($\lambda$ 0.55-1 $\micron$) grism spectra with a resolution R=80 per pixel for point sources. Data reduction for the WISP survey is performed using the \textit{AXe} software Cookbook, \citealt{Kuntschner2013}, \citealt{Kummel2009} but 3D-HST Survey have developed a different reduction pipeline (\citealt{Momcheva2016}, \citealt{Brammer2012}).

%how a limiting magnitude is calculated
For each field in WISP and each pointing 3D-HST,  we compute a limiting magnitude by measuring the peak of the F110W or F140W or F160W magnitude probability distribution function. This probability distribution is estimated by fitting a kernel-density estimator (\citealt{parzen1962}, \citealt{rosenblatt1956}) to the magnitude histogram using Silverman's (\citealt{silverman1986}) rule for bandwidth selection. These limiting magnitudes as well the total area of each field are also reported in Table \ref{tbl:obs}


\section{Selection of UCDs }

\subsection{Signal-to-noise Ratios and $ \chi^ 2$  Distribution}
% what I did 
% I computed snr for all objects in all fields 
We computed a signal-to-noise ratio in the region 1.2$\micron$ $\textless$ $\lambda$ $\textless$1.35 $\micron$ and 1.51$\micron$ $\textless$ $\lambda$ $\textless$1.6 $\micron$  of all available [insert number] WF3 spectra from both surveys, avoiding the noisier part of the spectrum.
% I computed chi-square compared to spectral standards and fitting to a line the f-test statistic, made a cutoff of both snr and f-test
We then compared each spectrum to M7 to T9 SpeX spectral standards (\citealt{2005ARA&A..43..195K}), \citealt{2006ApJ...637.1067B}) and measure $\chi^2$. 
We also compared the same spectra to a straight line and measured a $\chi^2$. We used an F-test of with $\frac{F(\chi^2_{std}} {\chi^2_{line}} \textgreater 0.5$ to eliminate flat spectra from the sample. This reduces the number of spectra from [insert number] to [insert number]. We further reduce this number by cutting off low snr spectra with snr< 10.0, this further reduces the number of spectra from [insert number] to [insert number]

\subsection{Spectral Indices}
%definition of spectral indices
We use G141 spectra  ($\lambda$=1.11--1.67 $\micron$) to differentiate between the effects of FeH and \wat and \meth features. We choose regions in the near-infrared J and H bands (1.1--1.7 $\micron$) that sample \meth and \wat  molecular features observed in L and T-dwarfs. These regions are given by five wavelengths ranges: 1.15--1.20 \micron, 1.246--1.295 \micron, 1.38--1.43 \micron,  1.56--1.61 \micron, and 1.62--1.67 \micron; denoted by H$_2$O-1, J-Cont, H$_2$O-1, H-Cont, and \meth respectively (figure \ref{img:stds}). We compute indices by measuring median flux ratios between two wavelength ranges for all available WFC3 spectra in both surveys. 
%description of the spex sample
We also compute indices for SpeX spectra of M, L, T dwarf spectral standards (\citealt{2005ARA&A..43..195K}), subdwarf standards (\citealt{2007ApJ...669.1235L}); and a list of templates from the SpeX Prism Library (SPL, \citealt{2014arXiv1406.4887B}). These spectral templates consist of 2122 nearby M5-T9 low-resolution (R$\sim$75-120) near-infrared ($\lambda$= 0.8-2.5 $\micron$) spectra with signal-to-noise ratio $\textgreater$ 60. We report the spectral type \& signal to noise distribution for this sample in in figure \ref{img:spex}.
%selection criteri boxes, contamination and completeness
We define selection criteria using six  boxes in each of the 45 the index vs index spaces by fitting a line to the each of the  M5--L0, L0--L5, L5--T0, T0-T5, T5--T9 and subdwarf groups. The axis of the box is then determined  by the line, while width of the box is defined to be 1.5 $\times$ the scatter  of the templates from this line. We then select all WFC3 sources falling in these boxes. From these criteria, we define given a completeness and  contamination statistic as
\begin{equation}
    Contamination= \frac{S_{WFC3}}{WFC3}\times 100
\end{equation}

\begin{equation}
    Completeness= \frac{S_{SPL}}{SPL}\times 100
\end{equation} with $S_{WFC3}$ representing the number of selected sources in the field, $B_{WFC3}$ the number of selected sources that are M, L and T-dwarfs, and $WFC3$ the total number of sources in the field. In equation (3) $S_{SPL}$ denotes the number of selected SPL templates, and $SPL$ denotes the total number of spectra in SPL. 

Figures \ref{img:completeness}, \ref{img:contamination} and  Tables \ref{tbl:completeness}, \ref{tbl:contamination} report these contamination and completeness statistics for all selection criteria in each index-index space. We find that all criteria are at least 81\% complete to all spectral type ranges M5--T9; however, these selection criteria can be contaminated up to 79\%. We only use criteria with contamination less than 7\%, limited by the minimum contamination for  selecting L5-T0 dwarfs, thus selecting a total of  3272 sources. We  compare each of these spectra to spectral standards, and visually inspect the most promising candidates. We discuss these candidates in the next section.

\section{M, L, T Dwarfs Candidates}

We find 235 M, L, T dwarfs. All spectral types are given by the best fit standard spectrum obtained by comparing each WFC3 spectrum spectral standards using a $\chi^2$ statistic (section 3). We compute  a spectro-photometric distance for each M, L, T dwarf using their spectral type and F110W, F160W, and F140W apparent magnitudes\footnote{3D-HST F140W and F160W magnitudes are computed using fluxes in photometric catalogs in \citealt{Skelton2014} as -2.5  $log_{10}F+ 25$. }. We first obtain a 2MASS J-F110W (or F140W or F160W) color by convolving the spectral standard SpeX spectrum  for given the spectral type with 2MASS J and F140W (or F110W or F160W) filter profile (Figure \ref{img:filters}). This color is then used to determine the apparent 2MASS J magnitudes for the sources by subtracting their F110W (or F140W or F160W) apparent magnitudes. Absolute magnitudes are computed using absolute magnitude/spectral type relations from \citealt{2012ApJS..201...19D} valid for a spectral type range of M6-T9. All uncertainties are estimated by standard error propagation formulas. We report these distances and photometry of each M, L, T dwarf in Table \ref{tbl:dist}

Three of the T-dwarfs WISP 1232-0033 and WISP 0307-7243, WISP 1305-2538 were reported by \cite{2012ApJ...752L..14M} as  with spectral types T7, T4.5, and T9.5 respectively; and distances 270$\pm$60 pc, 400$\pm$60pc, 40--60 pc. \cite{2012ApJS..200...13B} reported  AEGIS 1418+5242 and GOODS as T6 and L4 dwarfs respectively. They estimated AEGIS 1418+5242  distance at 300--400 pc using absolute magnitude relations from \cite{2004AJ....127.2948V}.

[need to talk more about T-dwarfs]

\section{Discussion}
\subsection{Magnitude Completeness}
\subsection{Number Density and Comparison to Galactic models}

\acknowledgements

This work is based on observations taken by the 3D-HST treasury program (GO 12177 and 12328) with the NASA/ESA HST, which is operated by the Association of universities for Research in Astronomy, Inc. under NASA contract NAS5-26555.

%\software{Astropy}

%\newcommand{\figfolder}{/users/caganze/research/wisps/figures/}


\begin{figure*}
   \centering
   \includegraphics[scale=0.6]{\figfolder par1.pdf}
   \caption{Example of a reduced grism spectrum of WISPS-01}
   \label{fig:spexsample}
\end{figure*}


\begin{figure*}
   \centering
   \includegraphics[scale=0.8]{\figfolder spexsample.pdf}
   \caption{Calibration Samples of UCDs used in this study}
   \label{fig:spexsample}
\end{figure*}


\begin{figure*}
    \centering
    \includegraphics{\figfolder standards.pdf}
    \caption{M5-T9 low resolution SpeX spectral standards (\citealt{2010ApJS..190..100K}) with highlighted bands showing the definition of spectral indices used in this study}
    \label{fig:indexdefinition}
\end{figure*}



\begin{figure*}
    \centering
    \includegraphics[scale=0.8]{\figfolder fields_skymap.pdf}
    \caption{Sky map of all the pointings in WISPS and 3D-HST}
    \label{fig:skymap}
\end{figure*}

\begin{figure*}
    \centering
    \includegraphics[scale=0.8]{\figfolder f_test_snr_distr.pdf}
    \caption{f-test and SNR-J distributions of all Spectra in both surveys}
    \label{fig:skymap}
\end{figure*}



\begin{figure*}
    \centering
    \includegraphics[scale=0.75]{\figfolder filter_profiles.pdf}
    \caption{Comparison between different HST and 2MASS filters used in this study}
    \label{fig:filterprofiles}
\end{figure*}



\begin{figure*}
\centering
\includegraphics[scale=0.4]{\figfolder sensitivity_illustration.pdf}
\caption{ Example of 2 HST-3D spectra before and after continuum correction to obtain the correct slope. The sensitivity curve is plotted in grey.}
\label{fig:sensitivity}
\end{figure*}

\begin{figure*}
    \centering
    \includegraphics[scale=0.4]{\figfolder index_index_plots.pdf}
    \caption{Best selection criteria for different subtype ranges. Both the calibrartion samples and the contaminants are shown}
    \label{fig:indexplots}
\end{figure*}


\begin{figure*}
    \centering
    \includegraphics[scale=0.5]{\figfolder completeness_contamination.pdf}
    \caption{Visual Representation of CPT and COMP statistics for all possible combination of spectral indices for each subtype range. Although the overall completenesses of each box is high (\textgreater 80\%), the contamination may vary. We only use selection criteria with the lowest possible contamination, however, any comibination of these indices could be useful for selecting UCDS in other surveys  }
    \label{fig:completenesscontamination}

\end{figure*}


\begin{figure*}

  \centering
    \includegraphics[scale=0.5]{\figfolder confusion_matrix.pdf}
    \caption{Confusion matrix for the random forest classifier used in this survey}
    \label{fig:matrix}

\end{figure*}

\begin{figure*}
    \centering
    \includegraphics[scale=0.4]{\figfolder candidates.pdf}
    \caption{Spectral Sequence of UCDs Discovered in WISPS \& 3D-HST. The right plot shows the 1D spectrum where the shaded region is the reported contamination by the survey, the middle plot shows the WFC3 image acquired in either F140W, F160W or F110W filter and the far-left plot shows the cutoff of the G141 spectrum for that extracted object. The derived spectral type of each object is displayed in the left corner of the far-left plot}
    \label{fig:candidates}
\end{figure*}


\begin{figure*}
    \centering
    \includegraphics[scale=0.8]{\figfolder candidate_distances.pdf}
    \caption{Spatial distribution of the the UCD sample reported in this paper}
    \label{fig:candidedistances}
\end{figure*}

\begin{figure*}
    \centering
    \gridline{
    \fig{\figfolder mass_hubble_colors.pdf}{0.5\textwidth}{(a)}
    \fig{\figfolder hst_relations.pdf}{0.5\textwidth}{(b)}
    }
    %\includegraphics[scale=0.5]{\figfolder hst_relations.pdf}
    \caption{ (a) Offsets between 2MASS J, H magnitudes and HST F110W, F140W, F160W magnitudes as a function of spectral type (b) Absolute  magnitude-spectral type relations for HST and 2 MASS filters. For HST filters, the dotted green curve shows the derived relation using only the offset between the respective HST filter and 2MASS J filter while the blue curve shows the derived relation using the offset with the 2MASS H filter. The solid line shows a best-fit 6th-order polynomial used, considering the wavelength coverage of the respective filters (figure \ref{fig:filterprofiles}). We report the coefficients of these polynomials in table \ref{tab:polynomials}}
    \label{fig:candidedistances}
\end{figure*}

\begin{figure*}
    \centering
    \includegraphics[scale=0.6]{\figfolder snr_fits.pdf}
    \caption{Linear fits between SNR-J and apparent F110W, F140W, F160W magnitudes using the sample of UCDs. These relations are reported in table \ref{tab:polynomials} and used to estimate SNR-J for different apparent magnitudes}
    \label{fig:candidedistances}
\end{figure*}

\begin{figure*}
    \centering
    \includegraphics[scale=0.5]{\figfolder mag_limit.pdf}
    \caption{Magnitude distribution of point sources (solid lines) and all the sources (dotted lines) in both WISP \& 3D-HST. We estimate the limiting magnitudes based on the distribution of point sources. For wisps the limiting magnitudes are F110W=22.0, F140W= 21.5, and F160W= 21.5. For 3D-HST the limiting magnitudes are F140W=22.5 and F160W. These magnitudes are used to compute the effective volumes for each spectral type}
    \label{fig:maglimit}
\end{figure*}

\begin{figure*}
    \centering
    \includegraphics[scale=0.6]{\figfolder selection_function_samples.pdf}
    \caption{Visualization of our selection function as a function accross spectral type and SNR-J. The label "F-test" indicates spectra with F-test \textgreater 0.5, the label "RF" indicates the spectra labelled as UCDs by the random forest classifier, and the label "Indices" indicates the spectra selected by our best selection criteria. The bar indicates the selection probability defined as the number of spectra selected over the total number of spectra in each SNR-J, spectral type bin. In the Monte-Carlo simulation, we use the most-selective selection function. }
    \label{fig:candidedistances}
\end{figure*}

\begin{figure*}
    \centering
    \includegraphics{\figfolder distance_distribution.pdf}
    \caption{Distribution of randomly drawn distances in all WISPS \& 3D-HST pointings \ref{eq:distancerandom} }
    \label{fig:candidedistances}
\end{figure*}

\begin{figure*}
   \centering
   \includegraphics[scale=0.6]{\figfolder simulations_dists.pdf}
   \caption{Monte-Carlo simulation: distribution of randomly drawn masses, ages, distances and computed spectral types, SNR-J and apparent F140W following relations defined in this work }
  \label{fig:simulationdists}
\end{figure*}

\begin{figure*}
    \centering
    \includegraphics[scale=0.6]{\figfolder oberved_numbers.pdf}
    \caption{Comparison between the measured number densities and the expected number densities based on the Monte-Carlo simulation based on different age distirbutions. We observe an under-prediction within Poisson-like errobars ($\sim \sqrt N$) between the predictions and the measurements per subytype but an agreement within subgroupings of 5 subtypes. The total number of predicted UCDs with spectral types $\geq$ L0 is 16.9 while the total number of UCDs within the limiting magnitudes (F110W \textgreater22.0 or F140W \textgreater21.5) is 16}
    \label{fig:simulationnbrs}
\end{figure*}

%print (3*y+x)
%x=0, 1, 2

%\begin{figure}\ContinuedFloat
%    {\foreach\y in {0,1,...,8}{%
%        \gridline{\fig{\figfolder /ltwarfs/spectrum\number\numexpr   1+ 2*\y\relax.pdf}{0.3\textwidth}{(\number\numexpr  1+2*\y)}
 %           \fig{\figfolder /ltwarfs/spectrum\number\numexpr  2 + 2*\y\relax.pdf}{0.3\textwidth}{(\number\numexpr  2 + 2*\y)}
            %\fig{\figfolder /ltwarfs/spectrum\number\numexpr  3 + 3*\y\relax.pdf}{0.3\textwidth}{(\number\numexpr  3 + 3*\y)}}
            %\fig{\figfolder /ltwarfs/spectrum\number\numexpr  12 + \y\relax.pdf}{0.2\textwidth}{(\number\numexpr  12 + \y)}}
%            }}

%\end{figure}


%\begin{deluxetable*}{cccccccc}
\tablecaption{ Best Selection Criteria \label{tab:criteria}}
\tablehead{ \colhead{SpT Range} & \colhead{X-axis} & \colhead{Y-axis} & 
\colhead{m} & \colhead{b} & \colhead{s} &  \colhead{Completeness} & \colhead{Contamination} }
\startdata L0--L5 & H-cont_/\wat-1 & \meth_/J-cont & 0.0& 1.0 & 0.6 & 0.99 & 0.14 \\
L5--T0 & H-cont_/\wat-1 & \wat-2_/J-cont & 0.17 & 0.29 & 0.39, & 0.95 & 0.165\\
T0--T5 & \meth_/\wat-1& \wat-2_/J-cont  & -0.1 & 0.28 & 0.43 & 0.93 & 0.135 \\
Y dwarfs & \wat-2_/J-cont &  \meth_/H-cont & 1.21 & 0.09  & 0.19 & 0.89 & 0.003 \\
T5--T9 &  \wat-1_/J-cont &  \meth_/H-cont & 1.61 & 0.03 & 0.09 &  0.95 & 0.003 \\
Subdwarfs & H-cont_/J-cont & \meth_/J-cont & 1.05 & -0.01 & 0.04 & 0.9 & 0.031 \\\enddata
\end{deluxetable*}

\clearpage
%\begin{rotatetable}
\begin{deluxetable}{ccccccchchchcc}
\tabletypesize{\scriptsize}
\tablecaption{ List of L0-T9 UCDs \label{tab:sample}}
\tablehead{\colhead{Short Name} & 
           \colhead{Grism ID} & \colhead{SNR-J} & \colhead{SpT} &\colhead{RA }& 
           \colhead{DEC} &  \colhead{F110W}& \nocolhead{F110Wer} &  
           \colhead{F140W} &  \nocolhead{F140Wer} & 
           \colhead{F160W} &\nocolhead{F160Wer } &\colhead{Distance (pc)} & 
           \colhead{Distance er} }
\startdata WISP0927+6027&PAR21-00005&324&L0&141.989319&60.462970&&&18.6&0.0&&&323&7\\
UDS0217-0509&UDS-25-G141\_36758&31&L1&34.318333&-5.153692&&&21.3&0.0&21.0&0.0&1086&81\\
WISP1154+1941&PAR338-00035&13&L1&178.716644&19.684700&22.2&0.0&&&21.9&0.0&1242&532\\
WISP1133+0328&PAR27-00036&10&L1&173.274353&3.477643&21.6&0.0&22.0&0.0&21.4&0.0&1106&407\\
WISP1618+3340&PAR65-00035&19&L1&244.707458&33.671520&21.7&0.0&&&21.3&0.0&933&390\\
WISP1150-2033&PAR199-00009&57&L1&177.706833&-20.561000&&&19.2&0.0&&&379&7\\
WISP1004+5258&PAR438-00051&10&L1&151.204559&52.974800&&&22.6&0.0&&&1845&32\\
WISP0015-7955&PAR244-00072&6&L1&3.785810&-79.930220&&&22.2&0.0&&&1531&27\\
GOODSS0333-2751&GOODSS-28-G141\_10859&34&L1&53.267498&-27.860249&&&21.4&0.0&21.3&0.0&1176&154\\
GOODSN1236+6211&GOODSN-33-G141\_09283&12&L2&189.223923&62.188259&&&22.2&0.0&22.0&0.0&1491&130\\
WISP1154+1939&PAR338-00136&4&L3&178.720154&19.660000&24.1&0.0&&&23.1&0.0&1809&661\\
WISP1625+5721&PAR156-00041&19&L4&246.353882&57.357600&&&&&21.4&0.0&1029&14\\
GOODSN1236+6214&GOODSN-24-G141\_21552&19&L4&189.161880&62.247669&&&22.0&0.0&21.8&0.0&1144&125\\
WISP1124+4202&PAR106-00047&11&L8&171.034760&42.042900&&&&&21.5&0.0&644&14\\
WISP1003+2854&PAR191-00077&6&T0&150.918884&28.912800&&&&&23.1&0.0&996&22\\
UDS0217-0514&UDS-12-G141\_10759&9&T0&34.435657&-5.240000&&&25.2&0.2&25.2&0.1&2401&352\\
WISP0326-1643&PAR467-00135&3&T1&51.511295&-16.722500&23.9&0.0&&&23.9&0.1&865&444\\
WISP1019+2743&PAR201-00044&4&T1&154.888565&27.720400&&&22.4&0.0&&&489&7\\
WISP0437-1106&PAR463-00176&4&T3&69.490608&-11.104400&24.3&0.1&&&24.3&0.1&850&380\\
GOODSS0332-2749&GOODSS-04-G141\_17402&13&T4&53.161709&-27.831562&&&22.6&0.0&22.9&0.0&501&69\\
WISP0307-7243&PAR130-00092&12&T4&46.921608&-72.732600&&&&&22.7&0.0&518&11\\
AEGIS1418+5242&AEGIS-03-G141\_17053&21&T4&214.710007&52.716480&&&22.7&0.0&23.1&0.0&540&78\\
GOODSS0332-2741&GOODSS-01-G141\_45889&31&T6&53.242542&-27.695446&&&22.1&0.0&22.9&0.0&289&52\\
WISP1232-0033&PAR58-00112&11&T7&188.176712&-0.551850&&&23.1&0.0&&&218&25\\
WISP1305-2538&PAR32-00075&11&T9&196.356232&-25.641300&23.1&0.0&23.0&0.0&22.7&0.1&359&495\\\enddata
\end{deluxetable}

 
 
         
\begin{deluxetable*}{ccccccccccc}
\tablecaption{ Polynomial relations used in this work where the polynomial is given by $y=\sum _{n=1} ^{7} c_n x^n$  \label{tab:polynomials}}
\tablehead{\colhead{ x} & \colhead{y} & \colhead{Scatter} & \multicolumn{7}{c}{Coefficients} \\ \hline \colhead{} & \colhead{} & \colhead{} & \colhead{c7} & \colhead{c6} &\colhead{c5} & \colhead{c4} & \colhead{c3} & \colhead{c2} & \colhead{c1} }

\startdata SpT & Abs F110W&0.4& -3.2  $\times$ 10$^{-6}$ &   5 $\times$ 10$^{-4}$& -3.4 $\times$ 10$^{-2}$  &1.15&
        -2.2 $\times$ 10$^{1}$ & 2.2 $\times$ 10$^{2}$ & -9 $\times$ 10$^{2}$ \\
SpT & Abs F140W&0.4&  3.5 $\times$ 10$^{-6}$ & -5.4 $\times$ 10$^{-4}$&  3.4  $\times$ 10$^{-2}$  & -1.1 &
         2  $\times$ 10$^{1}$&  -2 $\times$ 10$^{2}$ & 7.85 $\times$ 10$^{2}$ \\
SpT & Abs F160W&0.4&  3 $\times$ 10$^{-6}$ & -4.6 $\times$ 10$^{-4}$&  2.9  $\times$ 10$^{-2}$  & -0.96 &
         1.8  $\times$ 10$^{1}$&  -1.7 $\times$ 10$^{2}$ & 6.7 $\times$ 10$^{2}$ \\
F110W & $\log$ SNR-J &0.42& & & & & &-0.25& 6.63\\
F140W & $\log$ SNR-J &0.46 & & & & & &-0.27& 7.1\\
F160W & $\log$ SNR-J &0.45 & & & & & & -0.25& 6.69\\ 
 \enddata

\end{deluxetable*}




\begin{deluxetable*}{cccc}
\tabletypesize{\footnotesize}
\tablewidth{0pt}

 \tablecaption{ Galaxy Model Parameters \label{tab:model}}

 \tablehead{
 \colhead{Parameter} & \colhead{Description} & \colhead{Value} & \colhead{Reference}  }

\startdata $\rho_0$ & local UCD population &0.0055 pc$^{-3}$ & \cite{1999ApJ...521..613R} \\
R$_\sun$ &radial coordinate of the sun & 8000 pc &\cite{2008ApJ...673..864J} \\
Z$_\sun$& z coordinate of the sun & 25 pc&\cite{2008ApJ...673..864J}\\
H$_{thick}$& thick disk scale height& 900 pc&\cite{2008ApJ...673..864J} \\
H$_{thin}$& thin disk scale height &  300 pc &\cite{2008ApJ...673..864J}\\
L$_{thick}$&thick disk scale length & 3600 pc& \cite{2008ApJ...673..864J}\\
L$_{thin}$& thin disk scale length& 2600 pc &\cite{2008ApJ...673..864J}\\
p & halo power law index & 2.77&\cite{2008ApJ...673..864J} \\
q & halo sphere flattening parameter&0.64 & \cite{2008ApJ...673..864J} \\ 
f$_{thick}$ &relative number of thick disk stars &0.12 & \cite{2008ApJ...673..864J}\\
f$_{halo}$ &relative number of halo stars &0.0051 & \cite{2008ApJ...673..864J}\\ \enddata
 
 \vspace{-0.5cm}
\clearpage
\end{deluxetable*}


\begin{deluxetable*}{ccccc}
\tabletypesize{\footnotesize}
\tablewidth{0pt}
 \tablecaption{ Number Densities. Nex1 is the expected number with a uniform age distribution without selection effects, Nex2 is the expected number with a uniform age distribution taking selection effects into account and Nobs is the number in UCDs with that spectral type in the sample \label{tab:numbers}}
 \tablehead{
 \colhead{SpT} & \colhead{Volume (pc$^3$)} & \colhead{Nex1} & \colhead{Nex2} & \colhead{Nobs}   }
 \startdata L0&13399.0&22.1&4.1&8\\
L1&10935.0&17.0&8.9&1\\
L2&8859.0&11.6&6.4&0\\
L3&7177.0&9.4&4.6&4\\
L4&5840.0&7.3&4.3&0\\
L5&4753.0&5.4&3.4&0\\
L6&3828.0&4.3&2.0&0\\
L7&3025.0&3.3&1.0&1\\
L8&2346.0&2.5&0.8&0\\
L9&1806.0&1.6&0.7&1\\
T0&1411.0&1.0&0.6&0\\
T1&1142.0&0.8&0.5&0\\
T2&969.0&0.9&0.5&0\\
T3&854.0&1.1&0.5&1\\
T4&753.0&1.5&0.7&0\\
T5&611.0&1.9&1.0&1\\
T6&400.0&2.0&1.1&1\\ \enddata
 \vspace{-0.5cm}
\end{deluxetable*}

\bibliographystyle{apj} 
\bibliography{library}
\end{document}


