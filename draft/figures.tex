\newcommand{\figfolder}{/users/caganze/research/wisps/figures/}


\begin{figure*}
   \centering
   \includegraphics[scale=0.6]{\figfolder par1.pdf}
   \caption{Example of a reduced grism spectrum of WISPS-01}
   \label{fig:spexsample}
\end{figure*}


\begin{figure*}
   \centering
   \includegraphics[scale=0.8]{\figfolder spexsample.pdf}
   \caption{Calibration Samples of UCDs used in this study}
   \label{fig:spexsample}
\end{figure*}


\begin{figure*}
    \centering
    \includegraphics{\figfolder standards.pdf}
    \caption{M5-T9 low resolution SpeX spectral standards (\citealt{2010ApJS..190..100K}) with highlighted bands showing the definition of spectral indices used in this study}
    \label{fig:indexdefinition}
\end{figure*}



\begin{figure*}
    \centering
    \includegraphics[scale=0.8]{\figfolder fields_skymap.pdf}
    \caption{Sky map of all the pointings in WISPS and 3D-HST}
    \label{fig:skymap}
\end{figure*}

\begin{figure*}
    \centering
    \includegraphics[scale=0.8]{\figfolder f_test_snr_distr.pdf}
    \caption{f-test and SNR-J distributions of all Spectra in both surveys}
    \label{fig:skymap}
\end{figure*}



\begin{figure*}
    \centering
    \includegraphics[scale=0.75]{\figfolder filter_profiles.pdf}
    \caption{Comparison between different HST and 2MASS filters used in this study}
    \label{fig:filterprofiles}
\end{figure*}



\begin{figure*}
\centering
\includegraphics[scale=0.4]{\figfolder sensitivity_illustration.pdf}
\caption{ Example of 2 HST-3D spectra before and after continuum correction to obtain the correct slope. The sensitivity curve is plotted in grey.}
\label{fig:sensitivity}
\end{figure*}

\begin{figure*}
    \centering
    \includegraphics[scale=0.4]{\figfolder index_index_plots.pdf}
    \caption{Best selection criteria for different subtype ranges. Both the calibrartion samples and the contaminants are shown}
    \label{fig:indexplots}
\end{figure*}


\begin{figure*}
    \centering
    \includegraphics[scale=0.5]{\figfolder completeness_contamination.pdf}
    \caption{Visual Representation of CPT and COMP statistics for all possible combination of spectral indices for each subtype range. Although the overall completenesses of each box is high (\textgreater 80\%), the contamination may vary. We only use selection criteria with the lowest possible contamination, however, any comibination of these indices could be useful for selecting UCDS in other surveys  }
    \label{fig:completenesscontamination}

\end{figure*}


\begin{figure*}

  \centering
    \includegraphics[scale=0.5]{\figfolder confusion_matrix.pdf}
    \caption{Confusion matrix for the random forest classifier used in this survey}
    \label{fig:matrix}

\end{figure*}

\begin{figure*}
    \centering
    \includegraphics[scale=0.4]{\figfolder candidates.pdf}
    \caption{Spectral Sequence of UCDs Discovered in WISPS \& 3D-HST. The right plot shows the 1D spectrum where the shaded region is the reported contamination by the survey, the middle plot shows the WFC3 image acquired in either F140W, F160W or F110W filter and the far-left plot shows the cutoff of the G141 spectrum for that extracted object. The derived spectral type of each object is displayed in the left corner of the far-left plot}
    \label{fig:candidates}
\end{figure*}


\begin{figure*}
    \centering
    \includegraphics[scale=0.8]{\figfolder candidate_distances.pdf}
    \caption{Spatial distribution of the the UCD sample reported in this paper}
    \label{fig:candidedistances}
\end{figure*}

\begin{figure*}
    \centering
    \gridline{
    \fig{\figfolder mass_hubble_colors.pdf}{0.5\textwidth}{(a)}
    \fig{\figfolder hst_relations.pdf}{0.5\textwidth}{(b)}
    }
    %\includegraphics[scale=0.5]{\figfolder hst_relations.pdf}
    \caption{ (a) Offsets between 2MASS J, H magnitudes and HST F110W, F140W, F160W magnitudes as a function of spectral type (b) Absolute  magnitude-spectral type relations for HST and 2 MASS filters. For HST filters, the dotted green curve shows the derived relation using only the offset between the respective HST filter and 2MASS J filter while the blue curve shows the derived relation using the offset with the 2MASS H filter. The solid line shows a best-fit 6th-order polynomial used, considering the wavelength coverage of the respective filters (figure \ref{fig:filterprofiles}). We report the coefficients of these polynomials in table \ref{tab:polynomials}}
    \label{fig:candidedistances}
\end{figure*}

\begin{figure*}
    \centering
    \includegraphics[scale=0.6]{\figfolder snr_fits.pdf}
    \caption{Linear fits between SNR-J and apparent F110W, F140W, F160W magnitudes using the sample of UCDs. These relations are reported in table \ref{tab:polynomials} and used to estimate SNR-J for different apparent magnitudes}
    \label{fig:candidedistances}
\end{figure*}

\begin{figure*}
    \centering
    \includegraphics[scale=0.5]{\figfolder mag_limit.pdf}
    \caption{Magnitude distribution of point sources (solid lines) and all the sources (dotted lines) in both WISP \& 3D-HST. We estimate the limiting magnitudes based on the distribution of point sources. For wisps the limiting magnitudes are F110W=22.0, F140W= 21.5, and F160W= 21.5. For 3D-HST the limiting magnitudes are F140W=22.5 and F160W. These magnitudes are used to compute the effective volumes for each spectral type}
    \label{fig:maglimit}
\end{figure*}

\begin{figure*}
    \centering
    \includegraphics[scale=0.6]{\figfolder selection_function_samples.pdf}
    \caption{Visualization of our selection function as a function accross spectral type and SNR-J. The label "F-test" indicates spectra with F-test \textgreater 0.5, the label "RF" indicates the spectra labelled as UCDs by the random forest classifier, and the label "Indices" indicates the spectra selected by our best selection criteria. The bar indicates the selection probability defined as the number of spectra selected over the total number of spectra in each SNR-J, spectral type bin. In the Monte-Carlo simulation, we use the most-selective selection function. }
    \label{fig:candidedistances}
\end{figure*}

\begin{figure*}
    \centering
    \includegraphics{\figfolder distance_distribution.pdf}
    \caption{Distribution of randomly drawn distances in all WISPS \& 3D-HST pointings \ref{eq:distancerandom} }
    \label{fig:candidedistances}
\end{figure*}

\begin{figure*}
   \centering
   \includegraphics[scale=0.6]{\figfolder simulations_dists.pdf}
   \caption{Monte-Carlo simulation: distribution of randomly drawn masses, ages, distances and computed spectral types, SNR-J and apparent F140W following relations defined in this work }
  \label{fig:simulationdists}
\end{figure*}

\begin{figure*}
    \centering
    \includegraphics[scale=0.6]{\figfolder oberved_numbers.pdf}
    \caption{Comparison between the measured number densities and the expected number densities based on the Monte-Carlo simulation based on different age distirbutions. We observe an under-prediction within Poisson-like errobars ($\sim \sqrt N$) between the predictions and the measurements per subytype but an agreement within subgroupings of 5 subtypes. The total number of predicted UCDs with spectral types $\geq$ L0 is 16.9 while the total number of UCDs within the limiting magnitudes (F110W \textgreater22.0 or F140W \textgreater21.5) is 16}
    \label{fig:simulationnbrs}
\end{figure*}

%print (3*y+x)
%x=0, 1, 2

%\begin{figure}\ContinuedFloat
%    {\foreach\y in {0,1,...,8}{%
%        \gridline{\fig{\figfolder /ltwarfs/spectrum\number\numexpr   1+ 2*\y\relax.pdf}{0.3\textwidth}{(\number\numexpr  1+2*\y)}
 %           \fig{\figfolder /ltwarfs/spectrum\number\numexpr  2 + 2*\y\relax.pdf}{0.3\textwidth}{(\number\numexpr  2 + 2*\y)}
            %\fig{\figfolder /ltwarfs/spectrum\number\numexpr  3 + 3*\y\relax.pdf}{0.3\textwidth}{(\number\numexpr  3 + 3*\y)}}
            %\fig{\figfolder /ltwarfs/spectrum\number\numexpr  12 + \y\relax.pdf}{0.2\textwidth}{(\number\numexpr  12 + \y)}}
%            }}

%\end{figure}

