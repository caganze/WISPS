\documentclass[manuscript]{aastex63}

%\usepackage{natbib}
\usepackage{graphics}
\usepackage{amsmath}
\usepackage{ amssymb }

\usepackage{graphicx}
\usepackage{pgffor}
\usepackage{rotating}


\usepackage{pdftexcmds}

\usepackage{xcolor} 
\usepackage{pgffor}  
\usepackage{dpfloat}  

\usepackage{lipsum}

\begin{document}

\newcommand{\meth}{CH$_4$ }
\newcommand{\wat}{H$_2$O }

\newcommand{\indxmeth}{CH$_4$}
\newcommand{\indxwat}{H$_2$O}

\newcommand{\teff}{T$_{eff}$ }
\newcommand{\Msun}{M$_\sun$}
\newcommand{\chisquare}{$\chi^2$}

%%%%%%%%%%%%%%%%%%%%%%%%%%%%%%%%%%%%%%%%%%%%%%%%%%%%%%%%%%%%%%%%%%%%%%%%%%%%%%%%%%%%%%%%%%%%%%%%%%%%%%%%%%%%

%				              MAIN TEXT

%%%%%%%%%%%%%%%%%%%%%%%%%%%%%%%%%%%%%%%%%%%%%%%%%%%%%%%%%%%%%%%%%%%%%%%%%%%%%%%%%%%%%%%%%%%%%%%%%%%%%%%%%%%%


\title{Beyond the Local Volume: Surface Densities of Ultracool Dwarfs in Deep HST/WFC3 Parallel Fields }

\author{Christian Aganze}
\author[0000-0002-6523-9536]{Adam J. Burgasser}
\affiliation{Department of Physics, University of California, San Diego, CA 92093, USA}

\author{Mathew Malkan}
\affiliation{Department of Physics \& Astronomy, University of California,Los Angeles, CA 90095, USA }

\author{Chih-Chun Hsu}
\affiliation{Department of Physics, University of California, San Diego, CA 92093, USA}

\author[0000-0002-9807-5435]{Christopher A. Theissen}
\affiliation{Department of Physics, University of California, San Diego, CA 92093, USA}

\author{Daniella C. Bardalez Gagliuffi}
\affiliation{Department of Astrophysics, American Museum of Natural History, Central Park West at 79th Street, NY 10024, USA }

\author{Russell E. Ryan Jr}
\affiliation{Space Telescope Science Institute, 3700 San Martin Dr., Baltimore, MD 21218}
\author{Benne Holwerda}
\affiliation{Department of Physics and Astronomy, 102 Natural Science Building, University of Louisville, Louisville KY 40292, USA}


\begin{abstract}
Ultracool dwarf star and brown dwarfs provide a unique probe of large-scale Galactic structure and evolution, but until recently samples of sufficient size, depth and fidelity have been unavailable. Here, we present the identification of 193 M7-T9 ultracool dwarfs in 0.6~deg$^2$ of deep, low-resolution, near-infrared spectroscopic data obtained with the Hubble Space Telescope Wide Field Camera 3 instrument as part of the WFC3 Infrared Spectroscopic Parallel Survey and the 3D-HST survey with low-resolution near-infrared spectra.
We describe the methodology by which we isolate ultracool dwarf candidates from over 200,000 spectra, and show that selection by machine learning classification is superior to spectral index-based methods in terms of completeness and contamination. We use the spectra to accurately determine spectral types and spectrophotometric distances, the latter reaching to $\sim$2 kpc for L dwarfs and $\sim$ 400 pc for T dwarfs
%(UCDs) of the L, T, and Y spectral classes are the lowest-mass and coldest objects in the Milky Way. Like stars, they are tracers of Galactic structure and star-formation history, while the cooling of substellar UCDs provide additional probes for galactic archeology and chemical evolution. Wide-field optical and infrared surveys have uncovered thousands of UCDs, but primarily in the immediate solar neighborhood ($d<100$ pc). To push to larger distances, we have searched over 0.5 deg$^2$ of the WFC3 Infrared Spectroscopic Parallel Survey and the 3D-HST parallel survey with low-resolution near-infrared spectra. We report the discovery of 193 M7-T9 and T dwarfs with spectro-photometric distances up to $\sim$2 kpc for L dwarfs and $\sim$ 400 pc for T dwarfs. 
We model the luminosity function using population simulations that incorporate various assumptions of the underlying mass function and star formation history, accounting for stellar and substellar evolution, multiple stellar populations, and Galactic structure. Our star counts are generally consistent with a power-law mass function and constant star formation history for ultracool dwarfs, but with a vertical scale-height that varies with spectral type, ranging from 20--300 pc for L dwarfs to $\geq$400~pc for T dwarfs [WE NEED UNCERTAINTIES ON THESE NUMBERS; ARE THEY REALLY DISTINCT?]. 
This is consistent with prior simulation predictions that L-type brown dwarfs 
are on average a younger population.
%Future infrared sky surveys conducted with the James Webb Space Telescope (JWST) or the Euclid mission will put finer constraints on the luminosity of UCDs at large distances. We predict 
We use the same simulations to predict that the {\em Euclid} mission will recover $\sim 10^3-10^4$ L and T dwarfs spectra in the Euclid South and Euclid Fornax fields, providing sufficient statistics to use these low-mass stars and brown dwarfs to study Galactic structure and star formation history in detail.

\end{abstract}


\section{Introduction}
The structure and evolution of the Milky Way is largely informed from the spatial and kinematic distributions of its luminous stars, and area of study known as Galactic archeology \citep{1987ARA&A..25..603F,2012ARA&A..50..251I}. Large imaging and spectroscopic surveys such as the 
the Sloan Digital Sky Survey (SDSS; \citealt{2000AJ....120.1579Y}) have been critical for building our model of the Milky Way through 6-dimensional positions and velocities and detailed characterization of large-scale stellar populations. Overall star-count data have demonstrated that the primary structures of the Milky Way include 
a kinematically young population on concentric Galactic orbits, spatially distributed into one or more exponential disks (the thin and thick disk); 
a centrally concentrated population of metal-rich rich stars (the bulge), and
a widely-dispersed, kinematically older and metal-depleted population (the halo; e.g.\ \citealt{1978AJ.....83.1163D,1981ApJS...47..357B,2008ApJ...673..864J})
Analysis of the stellar evolutionary states of these populations suggest that the Galactic disk population has been continuously forming stars since 8--11 Gyr ago, while halo stars were largely formed 10--13 Gyr ago, containing stars formed early in the history of the universe \citep{1998ApJ...497..294L,2009ARA&A..47..371T,2013A&A...560A.109H}. 
More recently, the \textit{Gaia} astrometric mission \citep{,2018A&A...616A...1G}, combined with large-scale spectroscopic surveys such as 
LAMOST [SPELL OUT AND REF], APOGEE [SPELL OUT AND REF], GALAH [SPELL OUT AND REF], 
has greatly refined our picture of the Milky Way.
Notable discoveries from these surveys over the past few years include the 
inference of major merger events that likely formed the inner stellar halo and thick disk populations (Gaia-Enceladus/Gaia sausage: \citealt{2018Natur.563...85H,2018MNRAS.478..611B, 2018ApJ...856L..26M,Gallart_2019}; and the Sequoia event: \citealt{2018ApJ...856L..26M,2019MNRAS.488.1235M}); 
phase-space substructure and mixing among stars in the Solar Neighborhood, indicative of past perturbations of the Milky Way by satellite systems \citep{2018Natur.561..360A};
an ensemble of stellar streams that trace the tidal disruption of and stellar accretion from these satellites, and probe the Milky Way gravitational potential and dark matter profile \citep{2018MNRAS.479.2789B,2018MNRAS.481.3442M,2019arXiv190908924K};
and the detection of dozens of hypervelocity stars ejected through encounters with a central supermassive black holes in the Milky Way and Large Magellanic Cloud \citep{2018MNRAS.479.2789B,2019MNRAS.483.2007E}.
These results show the Milky Way to be a complex and dynamically evolving system.

All of these studies have focused on the brightest red giants and FGK Main Sequence stars, which provide both reach and reliability in the inference of stellar properties.
Ultracool dwarfs (UCDs; M $\lesssim$0.1\Msun, {\teff} $\lesssim$3000K; \citealt{2005ARA&A..43..195K}) provide an alternative, and potentially more enriching, population for studying the Milky Way system \citep{2004ApJS..155..191B,Ryan2017}. Ultracool dwarfs constitute $\sim$50\% of stars by number in the immediate Solar Neighborhood ($d < 100$~pc), and are abundant in every environment in the Galaxy \citep{1999ApJ...521..613R,2000ARA&A..38..337C,2007AJ....133..439C,2010AJ....139.2679B,2019ApJS..240...19K}. Stellar UCDs have lifetimes far in excess of the age of the Galaxy (\textgreater 10$^3$ Gyr, \citealt{1997ApJ...482..420L}), while substellar UCDs--brown dwarfs--do not fuse hydrogen and hence have effectively limitless lifetimes \citep{1962AJ.....67S.579K,1963ApJ...137.1121K,1963PThPh..30..460H}. Brown dwarfs also cool and dim as they age, developing distinct spectra shaped by strong molecular absorption features that are highly sensitive to temperature, surface gravity and metallicity. The thermal and chemical evolution of stellar and substellar UCDs provide potential age diagnostics that have already been exploited in stellar cluster studies \citep{1998ApJ...499L.199S,2005MNRAS.358...13J,2018ApJ...856...40M}
coeval binary systems \citep{2002ApJ...581L..43S,2009AJ....137.4621B}, 
amd searches of young moving groups near the Sun \citep{LopezSantiago2006,Gagne2015,Mamajek2015,Faherty2018}.

Ultracool dwarfs have historically been uncovered in wide-field red optical and infrared sky surveys.\footnote{Relevant UCD surveys include [SPELL OUT ALL OF THE SURVEY NAMES HERE]: 
DENIS: \citep{refId0,2003A&A...401..959P};
2MASS: \citep{2007AJ....133..439C, 2010ApJS..190..100K};
SDSS: \citep{2002AJ....123.3409H,2010AJ....139.1808S, 2014PASP..126..642S,2017AJ....153...92T};
UKIDSS: \citep{2013MNRAS.430.1171D,2013MNRAS.433..457B,Marocco01062015,2016A&A...589A..49S};
CFHT-LAS: \citep{Reyle2010a}; 
%Gaia: \citep{Reyle2018,2019AJ....157..231K}; 
VISTA: \citep{2012A&A...548A..53L,2014MNRAS.444.1793D}; 
Pan-STARRS: \citep{2011ApJ...740L..32L, 2013ApJ...777...84B, 2015ApJ...814..118B}; and 
WISE: \citep{2011ApJS..197...19K, 2016ApJS..224...36K, 2016ApJ...817..112S}.}
The intrinsic faintness of UCDs means that these surveys are generally limited to the immediate Solar Neighborhood (d $\leq$ 100~pc), and while they have enabled study of the local UCD luminosity function \citep{2007AJ....133..439C, 2008ApJ...676.1281M, Reyle2010a,2019ApJ...883..205B,2019ApJS..240...19K} they do not effectively probe Galactic structure or UCD halo populations, of which only a few are currently known \citep{2003ApJ...592.1186B,2008ApJ...681L..33L,2019MNRAS.486.1260Z}.
%that formed in the early metal-poor Galaxy which may have had a distinct initial mass function \citep{2002MNRAS.332L..65B,2003Natur.425..812B,2003ASPC..287..427B}. To investigate the complete UCD population of the Galaxy these scenarios, it necessary to identify UCDs populations beyond the solar neighborhood and further into the thick disk and halo of the Milky Way.

Deep, narrow-field imaging surveys provide one approach to investigating more distant ultracool dwarf populations. This work has been largely achieved by exploiting the sensitivity and imaging resolution of the {\em Hubble Space Telescope}, often in parallel with searches for high-redshift galaxies for which UCDs are ``contaminant'' population \citep{1996AJ....112.1472R}. A summary of existing deep surveys for UCDs is given in Table~\ref{tab:surveys}.
%Early work in this area includes measurement of M dwarf number counts in the {\em HST} Deep Field and Large Area Multi-Color Survey Groth Strip \cite{1997ApJ...482..913G,1997A&A...328....5K, 1997A&A...328...83C}. Analysis of these samples determined M dwarf thin and thick disk vertical scale heights of $\sim$325~pc and $\sim$650~pc, respectively, and ruled out very low mass stars as being an appreciable component ($<1$\%) of Galactic halo dark matter. 
%use star-count data in characterization of UCD populations beyond the local volume. A common approach is to use photometric selections cuts anchored to known sample. Early work by  conducted an M dwarf number counts to measure the halo luminosity function of the \textit{Hubble Space Telescope}'s Wide Field Camera 2 (HST-WFC2) and Planetary Camera (PC1) Deep Fields. They found 47 M dwarfs with M$_V$ \textgreater 13.5, and the distribution was consistent with a power law the mass function that turns at M $\sim$0.6 \Msun from $\alpha$=-1 to $\alpha$=0.44. Subsequent studies by \cite{} concluded that the contribution of low-mass stars (M$\sim$0.3 \Msun ) to the halo luminosity function is less than 1\%. 
One of the first studies to focus on distant UCDs was done by \cite{2005ApJ...631L.159R}, who identified 28 candidate L and T dwarfs in 135~arcmin$^2$ of deep imaging data obtained with the {\em HST} Advanced Camera for Surveys (ACS) instrument, selected by their \textit{$i-z$} colors to a limiting magnitude of $z < 25$. They  determined a thin disk vertical scale height of $\sim$350 pc, similar to prior measures of deep M dwarf counts \citep{1997ApJ...482..913G,1997A&A...328....5K, 1997A&A...328...83C}. 
\cite{Ryan2011} identified 17 candidate late M, L and T dwarfs in 232 arcmin$^2$ of {\em HST}/Wide Field Camera 3 (WFC3) imaging of the GOODS [SPELL OUT ACRONYM \& REF] using optical and near-infrared color selection, and determined a thin disk vertical scale height for these sources of 290$\pm$40~pc.
%Most recently, \cite{Vledder2016}, reanalyzing a sample of 274 M dwarfs from \cite{Holwerda2014} photometrically identified in 227 arcmin$^2$ of {\em HST}/WFC3 imaging data down to $F125W < 25$ from the Brightest of Re-ionizing Galaxies (BoRG, \citealt{2009ApJ...695.1591P}) survey, also derived a thin disk vertical scale height of $290^{+20}_{-19}$~pc and a local halo/disk M dwarf number ratio of 0.75\%$^{+0.25\%}_{−0.19\%}$, with no variation with M dwarf subtype. 
%(to a limiting magnitude of F125W=25)  identified 274 M dwarfs in  from the HST-WFC3  using optical and near-infrared colors and determining spectral types using $V-J$ color--M dwarf subtype relation (\citealt{2009ApJ...695.1591P}). They found a slightly higher density of M-dwarfs identified in the Northern fields compared to the Southern fields, and a  disk scale-height of 0.3--4 kpc with a dependence on subtype. The overall M-dwarf scale height was $\sim$600 pc, a number that is much larger than previous estimates mostly due to large uncertainties in the fit.  these data using a Markov Chain Monte Carlo method to fit the statistic to a galactic model including a thin disk, thick disk, and halo component. They derived a  and a central number density of $0.29^{+0.20}_{-0.13}$ pc$^{-3}$, with no correlation of model parameters with M-dwarf subtype,
Deep ground-based surveys have also identified samples of distant UCDs.
\citet{2010ApJ...723..184K} identified 7 late-L and T dwarfs in 9.3~deg$^2$ of optical and infrared imaging data from the Subaru Suprime-Cam Hawaii Quasar and T dwarf survey to a limiting magnitude of $z < 23.3$, spectroscopically confirming several of the targets and finding number counts to be consistent with a thin disk vertical scale height of $\sim$400~pc.
\citet{Sorahana2018} used the larger (130~deg$^2$) and deeper ($z < 24$) set of imaging survey data from the Hyper Suprime-Cam Subaru Strategic Program data [REF] to identify 3665 L dwarfs, and infer an average thin disk vertical scale height for these sources of 340--420 pc. 
Most recently, \cite{2019MNRAS.489.5301C} used data multi-band imaging data from the Dark Energy Survey (DES, [REF]), combined with photometry from wide-field imaging surveys, to identify 11,745 photometrically classified L0--T9 to a limiting magnitude of $z\leqslant22$, and estimate from these data a thin disk vertical scale height of $\sim$450~pc. 
These last two studies represent the largest compilations of UCDs to date, and use multiple colors to  segregate UCDs from other background sources.
However, all of these imaging surveys are still subject to contamination and inaccuracies in spectral classification that inhibit a detailed evaluation of completeness and variation on population parameters as a function of temperature or mass.
%another constraint on the number density of L dwarfs in the Galaxy using large samples (N\textgreater 10$^3$); however, as in many imaging surveys, poor accuracy in spectral types significantly affects the derived parameters.
%In addition to poor estimates of spectral types, these samples were contaminated with various non-stellar sources that could not be identified in the absence of spectral information. To push towards a larger and pure sample,  and consistent with previous studies. However, these studies do not probe statistics for later types. Recent work by  Ultimately, the large uncertainties on spectral types  of UCDs in imaging surveys poorly constrain their distances, and deep spectroscopic follow-up of these sources is not a priority for precious HST time. 


\begin{deluxetable}{lccccc}
\tabletypesize{\scriptsize}
\tablecaption{Deep Surveys for Ultracool Dwarfs.\tablenotemark{a}  \label{tab:surveys}}
\tablehead{
\colhead{} \\
\multicolumn{1}{c}{Survey Reference} &
\multicolumn{1}{c}{Area} &
\multicolumn{1}{c}{Limiting} &
\multicolumn{1}{c}{Limiting} &
\multicolumn{1}{c}{log$_{10}$ Effective} &
\multicolumn{1}{c}{Number} \\
\multicolumn{1}{c}{\& Methodology} &
\multicolumn{1}{c}{(deg$^2$)} &
\multicolumn{1}{c}{Magnitude} &
\multicolumn{1}{c}{Distance (pc)} &
\multicolumn{1}{c}{Volume (pc$^3$)\tablenotemark{b}} &
\multicolumn{1}{c}{Detected} 
}
\startdata 
\citet{2005ApJ...631L.159R} & 0.038 & $z < 25$ & 1250 (L0) & 3.1 (L0) & 28 LT \\
{\em HST}/ACS imaging &  &  & 250 (T0) & 1.6 (T0) & \\
\hline
\citet{2005ApJ...622..319P}  & 0.003 & $F775W < 27$ & 500 (L0) & 1.2 (L0) & 18 late-M \\
{\em HST}/ACS spectra &  &  & 170 (L5) & 0.1 (L5) & 2 L \\
\hline
\citet{2009ApJ...695.1591P}  & 0.028 & $z < 25$ & 1700 (M9) & 2.8 (M9) & 43 M4-M9 \\
{\em HST}/ACS spectra &  &  &  &  &  \\
\hline
\citet{Ryan2011}  & 0.064 & $F125W < 25.5$ & 3000 (L0) & 3.4 (L0) & 17 MLT \\
{\em HST}/WFC3 imaging &  & $F098M < 26.5$ & 700 (T0) & 2.5 (T0) & \\
\hline
\citet{2010ApJ...723..184K} & 9.3 & $z < 23.3$ & 570 (L0) & 4.8 (L0) & 7 LT \\
Suprime-Cam imaging &  &  & 120 (T0) & 3.1 (T0) & \\
\hline
\citet{2012ApJ...752L..14M}  & 0.2 & $F125W < 23$ & 400 (T0) & 2.8 (T0) & 3 T \\
{\em HST}/WFC3 spectra &  &  & 120 (T8) & 1.5 (T8) &  \\
\hline
\citet{Sorahana2018} & 130 & $z < 24$ & 900 (L0) & 6.2 (L0) & 3,665 L \\
Hyper Suprime-Cam imaging &  &  & 230 (L8) & 5.1 (L8) & \\
\hline
\citet{2019MNRAS.489.5301C} & 2,400 & $z < 22$ & 360 (L0) & 6.7 (L0) & 11,745 LT \\
Dark Energy Survey imaging &  &  & 65 (T0) & 4.8 (T0) & \\
\enddata
\tablenotetext{a}{Limiting distance $\gtrsim$400~pc $\approx$ $h_z$.}
\tablenotetext{b}{The effective volume of disk stars detectable in the survey, which accounts for their spatial distribution assuming an exponential disk with vertical scale height $h_z$ = 300~pc and radial scale length $d_R$ = 2600~pc, following \citet{2008ApJ...673..864J}.}
\tablecomments{[BE SURE TO CHECK THE COMPLETENESS OF THIS BY DOING A CITATION/REFERENCE SEARCH OF THESE SURVEYS]}
\end{deluxetable}


%Spectroscopic/grism surveys
A parallel approach to deep imaging is deep spectroscopy, which has been achieved by exploiting the slitless grism modes of the {\em HST} ACS and WFC3 instruments. These samples have greater fidelity in target identification (reduced contamination) and characterization, but are often shallower than imaging surveys.
%deep pencil beam samples of spectra in red optical and near infrared (NIR) with no prior selection of source type. NIR spectroscopy, in particular, samples the peak of UCD spectral energy distributions and measures broad molecular features that guide UCD classification schemes (\citealt{2005ARA&A..43..195K}). 
\citet{2005ApJ...622..319P} used {\em HST}/ACS red optical grism spectra of the
Hubble Ultra Deep Field (HUDF; [REF]) to a limiting magnitude of $i = 27$ to spectroscopically
identify 18 late M and 2 L dwarfs, and 
%in the and estimated their spectral types by fitting templates from \citet{Kirkpatrick2000} to their Gradient-Assisted Photon Echo Spectroscopy (GRAPES, ref) taken with the xxx instrument (ref). This study 
inferred a thin disk vertical scale height of 400$\pm$100 pc for these sources. 
\citet{2009ApJ...695.1591P} expanded this sample with {\em HST}/ACS slitless grism observations of the Probing Evolution And Reionization Spectroscopically fields (part of the Great Observatories Origins Deep Survey;  \citealt{Giavalisco2004}) to $z = 25$ and spectroscopically identified 43 M4--M9 dwarfs. 
Fitting a multi-component Galactic model, they inferred a thin disk vertical scale height of 370$^{+60}_{-65}$~pc, and estimated a thick disk vertical scale height of $\sim$1000~pc and halo/disk number ratio of 0.25\% for late M dwarfs.
Most recently, \citet{2012ApJ...752L..14M} spectroscopically identified 3 late T dwarfs in the WFC3 Infrared Spectroscopic Parallels Survey (WISPS; \citep{2010ApJ...723..104A}) based on the presence of strong {\meth} and {\wat} absorption features in 1.1--1.7~$\mu$m {\em HST}/WFC3 spectra. This small sample was sufficient to constrain the power-law index of the substellar mass function of the thin disk but not its vertical scale height.

In this paper, we present a new analysis of {\em HST}/WFC3 slitless grism spectra contained in both the WISPS and 3D-HST \citep{Momcheva2016,2012ApJS..200...13B,Skelton2014} surveys, expanding both the areal coverage and spectral types evaluated in these deep spectroscopic datasets.
In section~\ref{sec:data} we describe the survey data used.
In section~\ref{sec:selectionp} we describe our selection process, including 
and a robust analysis of two independent procedures using spectral indices and fitting:
selection by limiting ranges and by a random forest classified.
In section~\ref{sec:results} we review the properties of the XXX UCD identified in this sample, including
classifications and spectrophotometric distance estimates.
In section~\ref{sec:simulations} we compare our sample distribution to population simulations of UCDs
that take into account the underlying mass function, star formation history, thermal evolution
and Galactic spatial distribution of these source. 
We summarize our results in section~\ref{sec:summary}.

\section{Data}\label{sec:data}

We analyzed data from two surveys: 
the WFC3 Infrared Spectroscopic Parallel Survey \citep[WISPS;][]{2010ApJ...723..104A} and 
3D-HST \citep{Momcheva2016,2012ApJS..200...13B,Skelton2014}. These two surveys used the IR channel of the WFC3 camera \citep{doi:10.1117/12.789581}, providing low-resolution G102 ($\lambda$ = 0.8--1.17 $\micron$, $R \approx 210$) and G141 ($\lambda$ = 1.11--1.67 $\micron$, $R \approx 130$) grism spectra. 
Removal of the slit mask allows for overlapping spectra across the 136$\times$123 arcsec inner field of view of the WFC3 camera. Figure \ref{fig:par1} shows an WCF3 exposure of one of the WISPS fields.

\subsection{WISP survey data}
 WISPS was a 1000-orbit, {\em HST} pure-parallels survey covering 390 fields ($\sim$1500 arcmin$^2$), obtained concurrent with primary observations made using of the Cosmic Origins Spectrograph (COS) or Space Telescope Imaging Spectrograph (STIS). The goal of WISPS is to conduct a census of star-forming, high-redshift galaxies, and hence WISPS fields are typically at high Galactic latitudes ($|b| \gtrsim XXX\degr$).
 Grism data in G102 and G141 settings were obtained with an exposure time ratio of 2.4:1, although the individual exposure times varied according to the primary observations. [RANGE OF EXPOSURE TIMES?]
 %, while the exposure ratio of exposure times for imaging and grism is 6:1.
 Data were reduced using a combination of \texttt{AXe} \citep{Kuntschner2013, Kummel2009} and custom software, the latter to remove residual background and flag bad pixels. 
 %The typically higher Galactic latitudes of the WISPS fields resulted in less crowding compared to 3D-HST, 
 %The main sources of background are zodiacal light, and earth thermal emissions. Grisms spectra in WISPS have little crowding of the same fields given their high galactic latitudes, 
 Reference images were used to determine source location, and contamination corrections for overlapping spectra were computed from \texttt{AXe}.
 Direct images were also obtained broad-band F110W, F140W, and F160W filters, with source catalogs were generated using SExtractor \citep{1996A&AS..117..393B}. 
 We analyzed reduced grism and photometric data provided in WISPS release [NEED DATE/VERSION OF RELEASE] as reported in \cite{2010ApJ...723..104A}.  [IS THIS THE CORRECT REFERENCE FOR THE VERSION OF DATA WE ARE USING?]
 %Given the pure-parallel nature of this survey, the fields are observed in G102, G141 grism with no dithering between exposure. Reference images were also taken using F110W (corresponding to G102) and F140W (corresponding to G141) imaging cameras. To reach the same depth in both G102 and G141, 

% Data reduction and grism extraction was performed using a combination of \texttt{AXe} software \citep{Kuntschner2013, Kummel2009} and custom IDL pipelines to remove additional background and to flag bad pixels. The main sources of background are zodiacal light, and earth thermal emissions. Grisms spectra in WISPS have little crowding of the same fields given their high galactic latitudes, but multiple spectral orders do overlap. WISPS provides an estimate of contamination of each spectrum computed using \texttt{AXe} and source catalogs in WISPS were generated using SExtractor \citep{1996A&AS..117..393B}. We obtained WISPS G102 and G141 grism data as well as broad-band F110W, F140W, F160W photometric data and source catalogs from the Mukuliski Archive for Space Telescope (MAST\footnote{\url{https://archive.stsci.edu/prepds/wisp/}}). 

\subsection{3D-HST survey data}
%general description: wwhich fields
 3D-HST was an {\em HST} parallels survey of 248 orbits covering $\sim$600 arcmin$^2$ obtained during Cycles 18 and 19. 
 The goal of 3D-HST was to [WHAT??].
 This survey targeted four standard deep extra-galactic fields: 
 the All-wavelength Extended Groth Strip International Survey (AEGIS, \citealt{1538-4357-660-1-L1}), 
 the Cosmic Evolution Survey (COSMOS, \citealt{Scoville2007}), 
 the UKIDSS Ultra-Deep Survey (UKIDSS-UDS, \citealt{2007MNRAS.379.1599L}), 
 and the Great Observatories Origins Deep Survey (GOODS-South and GOODS-North \citealt{Giavalisco2004}),
 assembled under the umbrella of the Cosmic Assembly Near-infrared Deep Extragalactic Legacy Survey (CANDELS, \citealt{2011ApJS..197...35G,2011ApJS..197...36K}). [IS THIS RIGHT?]
 Spectral data were obtained using both the ACS/G800L [NEED WAVELENGTH RANGE, RESOLUTION] and WFC3/G141 grisms;
 we focused on the WFC3 data for this study.
 WFC3 exposures were obtained over two orbits, with typical exposure times of 5000~s per pointing. [IS THIS TRUE FOR ALL EXPOSURES OR IS THIS JUST THE MEAN? IF THE LATTER, WHAT WAS THE RANGE?]
Data were reduced using custom software, including generation of a contamination model, as described in \citet{Momcheva2016};
we used these data for our spectral analysis.\footnote{Data were retrieved from the survey's website \url{https://3dhst.research.yale.edu/Home.html} on [WHAT DATE??].}
Note that the extracted 1D spectra reported in \citet{Momcheva2016} are not continuum-corrected (Figure~\ref{fig:sensitivity}); we therefore applied the sensitivity curves provided in the 3D-HST data products to recover each spectrum's continuum shape.
 %The goal of 3D-HST was to obtain as the However, 3D-HST is only 70\% of the total footprint of the CANDELS. 
Given that standard fields were targeted in this survey, there is additional ground-based and space-based photometry 
from various other surveys in multiple optical and infrared bands, and we used the compilation of these provided in \cite{Skelton2014}.



%%observations and survey strategry
%  The pointings for 3D-HST are designed to cover CANDELS area, therefore there are additional ground-based and space-based photometry from various other surveys in several optical and infrared filters. 
%  Each pointing in 3D-HST is observed by two orbits using the G141 grism and the F140W filter with typical exposure times of 5000 s for G141 AND 800 s for F140W. Observations for most of the pointings in the survey were conducted from October 2010 to November 2012. However, the GOODS-North field is a part of the A Grism H-Alpha SpecTroscopic survey (AGHAST, GO-11600; PI: Wiener) and was observed between Sept. 16, 2009 and Sept. 26, 2010 and re-observed on April 19 and 24, 2011, due artifacts and background issue, with exposure times of 800 s for F140W AND 5200S in G141.  

%data reduction
% Data reduction in 3D-HST involves reducing both the direct F140W images and G141 grism images. The full description of the image reduction pipeline is described by \cite{Brammer2012}, \cite{Skelton2014}  and \cite{Momcheva2016}. Raw images were downloaded and passed through a pipeline that consists of removing satellite trails and artifacts through visual inspection, background -subtraction and flat-fielding. The main physical sources of time-dependent background are zodiacal continuum, scattered light and persistence from He emission at 1.083 micron. Both the reduction of the F140W and G141 images involved combining at most four dithered images. A standard method uses a drizzling algorithm implemented by the \texttt{AXe} software \citep{Kuntschner2013, Kummel2009}. However, drizzling is designed to work well for a large number of images. The shortcomings of this method inlude the introduction of correlated noise between adjacent pixels. To avoid these issues, 3D-HST stacked all the dithered images onto one grid, given that the dithered images are all separated by the same number of pixels by the design of the survey. In addition, reducing the grism images requires a reference image (different from the obtained F140W direct image) to generate a contamination global model of each pointing, to separate overlapping spectra and orders. The reference image was obtained by coming F125W, F140W, and F160W images of that pointing obtained from \cite{Skelton2014} data products, where the magnitudes of all objects in the fields are scaled to the F140W zero-point, and errors propagated. Based on the morphology and the magnitude of each source in the reference image, the full 2D-spectrum of each object was modeled from a 1D spectral energy distribution (SED). This contamination model was then used to correct for overlapping spectra and orders. These 2D-spectra of exactly 312 pixels each are then extracted. The reference image and the direct images are on the same grid, therefore no source matching was required for source identification.

% Data reduction of the 3D-HST WFC3 grism data is described in \citet{Momcheva2016}, and we use the data products from this study, in addition to the photometric catalog of \citet{Skelton2014} for our analysis \footnote{Data were retrieved from the survey's website \url{https://3dhst.research.yale.edu/Home.html} on [WHAT DATE??].}. This reduction involved a custom pipeline that accounted for source overlap in the generation of a contamination model. 
% The extracted 1D spectra reported in\citet{Momcheva2016} are not continuum-corrected (Figure~\ref{fig:sensitivity}); we therefore used the sensitivity curves provided in the 3D-HST data products to recover the continuum shapes. 
% We did not perform any additional reduction to the data.



\section{Selection of UCDs}\label{sec:selectionp}

\subsection{Spectral Calibration Sample}\label{sec:trainset}

To anchor our identification of new UCDs in the WISPS and 3D-HST datasets, we first created a spectral calibration sample composed of known UCDs with similar spectral coverage and resolution, drawn from nearly 3000 low-resolution ($\sim$75--120), near-infrared (0.9--2.5 $\micron$) spectra of nearby UCDs contained in the SpeX Prism Library  \citep[SPL\footnote{\url{https://cass.ucsd.edu/~ajb/browndwarfs/spexprism/library.html}};][]{2014arXiv1406.4887B}. 
We selected all spectra with median signal-to-noise S/N $>$ 10, consisting of 2,677 spectra. We then visually inspected these spectra to remove background contaminants (which are retained for our random forest classification as non-UCDs; see $\S$XXXX),
resulting in a sample of [XXX] M6 and later-type sources, which we refer to as the templates/SpeX sample. 
To these, we added 77 UCDs from \cite{Manjavacas2018} observed with the WFC3,
and 22 Y dwarf spectra obtained with the HST/WFC3 camera by \cite{Schneider2015}. 


\subsection{Pre-selection Constraints}

\subsubsection{Point-Source Selection}

We combined all grism and photometric data from the two surveys, a total of 271,915 sources.
%that have corresponding photometry in the provided photometric catalogs. 
To narrow down our selection, point sources in both surveys were identified using the stellarity index \texttt{CLASS\_STAR} $\neq$0 from generated SExtractor catalogs. 3D-HST provides an additional \texttt{star\_flag} flag for point sources based on F160W imaging data and the \texttt{FLUX\_RADIUS} flag, However, but we found that this flag eliminated 3 UCDs from the 3D-HST sample that were not rejected using the SExtractor stellarity index. This criterion reduced the sample down to 110,930 sources, or 40.7\% of the original spectral sample.

\subsubsection{J-band Signal-to-Noise Rejection}

To eliminate low S/N spectra that may be too ambiguous to identify or classify as UCDs, we performed a spectral S/N cut aimed to reduce selection bias between late-M, L, T and Y dwarfs. UCDs display strong \wat and \meth molecular absorption features in the $J$ and $H$ bands (1.1--1.6 \micron), so a S/N calculation that encompasses this full range will produce different results for different degrees of molecular absorption. We therefore defined a S/N ratio for the $J$-band continuum (hereafter $J$-SNR) spanning the range 1.2 $\micron$ $\leq$ $\lambda$ $\leq$ 1.3 $\micron$, which captures the continuum flux evenly across spectral types. We rejected the lowest S/N spectra by requiring $J$-SNR $\geq$ 3, which retained 46,370 spectra, or 17.1\% of the original spectral sample.
%/grisms, that is 38.7\% of the original point-source sample and 15.8\% of the total number of spectra including extended objects. We also measured the $J$-SNR for all the spectra in our calibration samples in a similar fashion.

\subsubsection{Spectral Template and Line Fitting}

A visual inspection of the remaining spectral data shows that common contaminants include emission line sources, featureless spectra with low S/N, spectra with absorption or emission features outside the primary \wat and \meth bands, and other artifacts. To further narrow down our selection of UCDs, we fit each WFC3 spectrum to the set of UCD spectral standards and to a straight line using a $\chi^2$ minimization method.
%following the method of \cite{2010ApJS..190..100K}. Hence, we obtained a spectral type classification for all point-sources. We also compared every spectrum to a straight line in the same wavelength region and measured $\chi^2$. 
These separate fits were aimed at distinguishing featureless spectra from UCDs spectra. 
The $\chi^2$ of the standard fits ($\chi^2 _T $) were computed as:
\begin{equation}
\chi^2_T= \sum _{\lambda = 1.15 \micron} ^{ 1.65 \micron} \frac{(\text{Sp}  - \alpha T)^2}{ \sigma ^2 }
\end{equation} 
where $\text{Sp}(\lambda)$ is the WFC3 spectrum, $\sigma$ the spectrum uncertainty, $T$ is the spectral template, and 
$\alpha$ a scale-factor that minimizes $\chi^2_T$: 
%\begin{equation}
%\alpha= \frac{\sum_{\lambda = 1.15 \micron}^{ 1.65 \micron} \frac{\text{Sp}\times\text{T}}{\sigma^2}}{\sum_{\lambda = 1.15 \micron}^{ 1.65 \micron} \frac{\text{T}^2}{\sigma^2}}.
%\end{equation} 
\begin{equation}
\alpha= \left(\sum_{\lambda = 1.15 \micron}^{ 1.65 \micron} \frac{\text{Sp}\times\text{T}}{\sigma^2}\right) / \left(\sum_{\lambda = 1.15 \micron}^{ 1.65 \micron} \frac{\text{T}^2}{\sigma^2}\right)
\end{equation} 
(cf.\ \citealt{2005ApJ...623.1115C}). The $\chi^2$ for the line fit ($\chi^2 _L$) was computed as:
\begin{equation}
\chi^2 _L  = \sum _{\lambda = 1.15 \micron} ^{ 1.65 \micron} \frac{(a + b \lambda-\text{Sp} )^2}{ \sigma ^2 }
\end{equation} 
where $a$ and $b$ are linear parameters determined through least-squares minimization. 

To distinguish between these fits, we used an $F$-test statistic to determine if the standard fit was a statistically significant improvement over the line fit. We  eliminated sources with $F$($\chi^2_T/\chi^2_L)$ $< 0.4$; i.e., where the probability of a UCD standard being a better fit to the spectrum than a line is less than 40\%. This selection cut retained 8,148 spectra, or 
%that is 18.9\% of point-sources with $J$-SNR $>3$, 7.3\% of all point-sources and 
3\% of the original spectral sample. This process also provided a first-order estimate of the spectral classification of any UCD candidates in our sample.

\subsection{Spectral Index Selection}

\subsubsection{Defining Spectral Indices}


Our final selection criteria was based on the measurement of spectral indices which sample the the strong \meth and \wat molecular features present in the $J$- and $H$-band regions. Spectral indices are commonly used to classify UCD spectra, and are typically tailored to the spectral resolution and region sampled \citep{1999AJ....117.1010T,2000AJ....119.3019C,2007ApJ...657..511A,2007ApJ...658..557B}. 
We selected five wavelength regions bracketing the primary absorption features (Table~\ref{tab:criteria}), measuring the median flux in these regions. Uncertainties for these band measurements were estimated by Monte Carlo sampling of the individual flux measurements assuming normal distributions  scaled by the spectral uncertainties. Ten index ratio were then defined from these band fluxes: 
 \begin{equation} 
 {\rm A/B} =\frac{ \langle  F(\lambda_{A,l}<\lambda < \lambda_{A,h}) \rangle }{  \langle F(\lambda_{B,l} < \lambda <\lambda_{B,h}) \rangle }
 \end{equation}
where $\langle{F}\rangle$ denotes the median flux value, and $\lambda_{A,lh}$ denotes the lower and upper wavelength limits for spectral band $A$. To select for Y dwarfs, which have extremely low levels of flux in the \wat and \meth absorption bands, we also defined an additional set of indices that combined the flux measurements in these bands, e.g., (\wat-1+\meth)/J-cont, (\wat-2+\meth)/H-cont, etc. 

These indices were measured on both the WFC3 data and our spectral calibration sample. The latter were used to determine selection criteria for UCD subtype groupings of M7--L0, L0--L5, L5--L0, T0--T5, T5--T9, Y dwarfs, and late-M/L subdwarfs. 
For each possible index-index pairing (45 combinations), we evaluated the
distribution of index measurements for calibration templates within each of these subtype groupings, and defined  parallelogram regions in which these templates clustered. The parallelograms were determined by first measuring a linear trend ($y = mx+b$) in the index-index data ($x,y$) for a given subtype group, then defining a perpendicular range about that line that encompassed 3 times the standard deviation (STD) of template values. This process produced four vertex points encompassing each template cluster in each of the possible index-index spaces (Table~\ref{tab:polynomials}). Note that for M and L dwarfs, we used rectangular boxes; i.e., assuming the slope $m$ = 0. [THIS IS AN ARBITRARY CHOICE, WHY DO THIS?]

%To define the parameters of each box, we fitted a characteristic line to each index pair ($x$-index, $y$-index) within a subtype, defining the slope/direction of the box: $y=m\times x-\mathrm{index}+b$. Each box has four vertices ($v_1$, $v_2$, $v_3$, $v_4$) . These vertices are computed as $v_1=$  (xmin, ymax), $v_2=$ (xmin, ymin), $v_3=$ (xmax, ymax), $v_4=$ (xmax, ymin), where (xmax, xmin) =  median($x$-index) $\pm$ 3 $\times$ std($x$-index). On the $x$-axis, if xmax is greater than the maximum of $x$-index, or if xmin is less than the minimum of the x-index, we set $x$-min and $x$-max and the minimum and maximum of $x$-index. The extent of the boxes on the $y$-axis are determined by $\text{(ymax, ymin)}= m \times \text{(xmax, xmin)} + b \pm 0.4 \times dy$, where $dy$ is the range of $y$-index (max($y$-index) $-$ min($y$-index)). This simple algorithm allows our selection boxes to enclose most of the objects in the subytpe, avoid outliers but not extend too far from away the defining sample. . 

\subsubsection{Completeness and Contamination}


To quantify the effectiveness of these selection regions to identify UCDs in a given subtype grouping and exclude contaminants, we defined two statistical metrics measuring completeness and contamination:
\begin{equation}
CP(SG,I) =\frac{N_{\rm T}^*(SG,I)}{N_{\rm T}(SG)}
\end{equation}
\begin{equation}
CT(I) = \frac{N_{\rm WFC3}^*(I)}{N_{\rm WFC3}}. 
\end{equation} 
In the first equation, $N_{\rm T}(SG)$ is the total number of templates in subtype group $SG$, while $N_{\rm T}^*(SG,I)$ is the number of templates within an index-index selection region $I$; CP = 1 constitutes selection of all templates.
In the second equation, $N_{\rm WFC3}$ = 8,148 is the total number of WFC3 spectra, while $N_{\rm WFC3}^*(I)$ is the number within an index-index selection region $I$. As our expected number of UCD discoveries is assumed to be much smaller than $N_{\rm WFC3}$, a high selection rate of WFC3 sources implies a high level of contamination; CT = 1 indicates a highly contaminated selection region. 

To maximize UCD selection (high CP) and minimize contamination (low CT), we rank-ordered for each spectral subgrouping those index-index pairs that had CP $\geq$ 0.9 and the lowest values of CT; these are listed in Table~\ref{tab:criteria}.
Indices measuring the strength of {\wat} absorption (\indxwat-1/$J$-Cont, \indxwat-2/\indxwat-1) were best for late-M and L dwarfs, and subdwarfs. 
Indices measuring the strength of {\meth} absorption (\indxmeth/$H$-Cont) were best for T dwarfs. 
Continuum ratios $H$-cont/$J$-Cont were best for subdwarfs and T dwarfs. 
The combined \wat and \meth indices (\indxwat-2+\indxmeth/$J$-cont \indxwat-1+\indxmeth/$H$-cont) were best for the Y dwarfs.
%The best selection indices for each subtype were found to be: (\indxwat-1/$J$-Cont,  \indxwat-2/\indxwat-1) for L0--L5, measuring the strength of \wat relative to the $J$-continuum, (  \indxwat-1/$J$-Cont, \indxmeth/$H$-Cont) for L5--T0 measuring the strength of \wat and \meth relative to both the $J$ and $H$-continua, (\indxwat-1/$J$-Cont, \indxmeth/\indxwat-1) for M7--L0 measuring the strength of \wat and \meth relative to the $J$-continuum, (\indxwat-2/$J$-Cont, $H$-cont/$J$-Cont) for T0--T5 subtypes measuring the strength of \wat in the $H$-band and \meth relative to both continua,  (\indxwat-2/$J$-Cont, \indxmeth/$H$-Cont) for T5--T9 measuring the strength of \meth relative to both continua, and (\indxwat-1/$J$-Cont, $H$-cont/$J$-Cont) for subdwarfs measuring the strength of \wat and \meth relative to the $J$-continuum. 
%There are 45 possible combinations of these 10 index ratios but we chose these for their low contamination rates. Finally, to select Y dwarfs, as stated, we added the strength of \wat and \meth in both bands given the flux in these regions for Y dwarfs is extremely weak. The index of choice is (\indxwat-2+\indxmeth/$J$-cont \indxwat-1+\indxmeth/$H$-cont) for Y dwarfs. 
We applied the optimal index-index selection criteria to the 8,148 WFC3 spectra, selecting 2,174 (0.8\% of the initial spectral sample) in the following subtype groups: M7--L0: 1,413; L0--L5: 530, L5--T0: 437; T0--T5: 160; T5--T9: 14; Y dwarfs: 19; and subdwarfs = 314. The greatest degree of contamination is in the M7--L0 subtype group, due to the relative weakness of absorption features for these earlier spectral types. 

\subsubsection{Visual Selection}

After all of the preceding selection criteria were applied, we visually inspected the remaining spectra to confirm their UCD nature and spectral type, the latter based on both the template fitting and spectral indices). Keeping only those sources whose spectra visually matched a spectral template, we identified a total of [XXXX] confirmed UCDs, distributed as listed in Table~\ref{tab:????}.

Retrospectively, we estimated the false positive rates (FP) for each subtype group as 
\begin{equation}
FP(SG) = 1-\frac{N_{WFC3}^T(SG)}{N_{WFC3}^*(SG)}
\end{equation} 
where $N_{WFC3}^T$ is the number of WFC3 spectra confirmed as a true UCD. Our best selection criteria have FP values between 70--100\%, indicating significant contamination remains after the selection steps, necessitating visual inspection.  
Nevertheless, the number of spectra for which this inspection  is necessary was reduced by over a factor of 100 by the various selection criteria. 


\subsection{Selection by Random Forest Classifier}

%justification
As an alternative to using 2D boxes in index-index space to select UCD candidates, we separately trained a random forest classifier to determine which of the WFC3 spectra were likely UCDs.
Random forest algorithms use a set of measured features and random, independent decision trees to determine classification labels on a set of objects. The forests are trained using a well- and consistently-characterized training set where the classification labels are already known.  Random forests are a practical method to classify large datasets, given that the algorithm is relatively fast and unbiased by noise [NEED REFERENCE FOR THIS ASSERTION]. [ALSO NEED A STATEMENT AS TO WHEN THESE ARE SUPERIOR TO OTHER MACHINE LEARNING METHODS, SUCH AS NEURAL NETS OR MCMC]
Random forests have been shown to reliably predict M-dwarf subytpes based on their photometric colors \citep{2019arXiv190505900H}, and perform star-galaxy photometric classification in transient surveys \citep{2017AJ....153...73M}. 

%training set
\subsubsection{ Training Set and Classification Features}

We used the \texttt{RandomForestClassifier} implementation of the random forest algorithm provided in the \texttt{scikit-learn} package \citep[IS THERE AN UPDATED REFERENCE FOR THIS?]{2012arXiv1201.0490P}. [NOTE: THE FOLLOWING IS NOT CONSISTENT WITH THE SPECTRAL CALIBRATION SAMPLE DESCRIBED ABOVE, PLEASE BE SURE THESE ARE DESIGNED CONSISTENTLY]
 Our a training set includes 2,677 known UCDs from the spectral calibration sample, and 3,612 spectra of contaminants based on visually-rejected sources in the WFC3 sample [AFTER ALL THE OTHER SELECTION CRITERIA ARE APPLIED?] [DOES THIS INCLUDE MANJAVACAS AND SCHENIDER SAMPLES? WHY ONLY 19 OF SCHNEIDER? WHAT ABOUT CONFIRMED WFC3 UCDS?]. 
 %, including 19 objects from the Schneider set. 
 The former are labeled UCDs, and include spectral types M6 and later; the latter are labeled non-UCDs. 
 %these sources using two labels: UCDs, which are objects with spectral types $\geq$ M6, and non-UCDs which are objects with spectral types $\leq$ M6 and/or part of the set of visually confirmed contaminants. Although the difference between an M6 UCD and M7 UCD is not as rigid, we make this cut to reduced biases in selection objects with subtypes $\geq$ M7. 
% In addition, all objects in the training set have a $J$-SNR $\geq 3$. 
 [I'M CONFUSED ON HOW THERE ARE THEN A DIFFERENT SET OF NUMBERS AFTER THIS]. In total, we use 4,009 UCDs and 2,299 non-UCDs in our training sample.
 
%features and results
Choosing an appropriate set of features is a critical part of designing a good machine learning. We used the spectral indices defined above, and the statistics $J$-SNR, $\chi^2_L$, $\chi^2 _T$, and the $F$-test statistic comparing the last two values. [ISN'T THE F-TEST STATISTIC REDUNDANT? WHAT PURPOSE DOES J-SNR SERVE?]. For source for which these features are not measureable [WHY WOULD THIS BE?], we assigned values of -99999.9, then scaled all features in the range [0, 1] using \texttt{MinMaxScaler}. [THIS IS TOTALLY NOT GOING TO WORK - IF YOU ASSIGN A HUGE NEGATIVE VALUE, ALL OTHER VALUES ARE EFFECTIVELY SQUEEZED TO BE EXACTLY ONE. YOU NEED TO FIGURE OUT A DIFFERENT WAY TO DEAL WITH MISSING DATA]. [NEED SOME DETAIL HERE ON HOW THE DECISION NODES ARE SET - ARE THEY SETTING AN ARBITRARY SPLIT VALUE BETWEEN 0 AND 1? HOW DOES IT DECIDE WHAT FEATURE TO TEST? IN WHAT ORDER?] The tree is then trained on the training set [EXPLAIN WHAT THE TRAINING IS DOING - IS IT ELIMINATING DECISION TREES THAT GIVE THE WRONG ANSWER? NEED MORE DETAIL HERE]

To test the accuracy of our classifier, we used the two-fold cross-validation score (CV). We split the training sample into two randomly-assigned 50\% partitions, trained the random forest on one partition and then used this to classify the second partition (see \citealt{2017AJ....153...73M}). We obtained a CV accuracy score of 97.6\% [IS THIS GOOD? WHAT IS EXPECTED? HOW DO WE INTERPRET THIS NUMBER? ALSO, HOW MANY DEFINITE UCDS WERE REJECTED, VERSUS NON-UCDS SELECTED? WE NEED TO SEE THE FALSE POSITIVE AND FALSE NEGATIVE SCORES FROM THIS]. 

\subsubsection{Results}

We then trained the random forest on the entire training set and applied it to the 110,930 WFC3 spectra of point-source objects with $J$-SNR $\geq3$ [NOTE: IN TEXT ABOVE THIS NUMBER IS 46,370]. This process classified 78 sources as UCDs. Visual inspection confirmed 58 of these, implying a false positive rate FP = 26\%. This is far superior in efficiency than the spectral index selection. [NEED TO ALSO COMPUTE FALSE NEGATIVE - HOW MANY VISUALLY SELECTED SOURCES WERE NOT SELECTED BY RANDOM FOREST, AND HOW MANY SOURCES SELECTED BY RANDOM FOREST WERE NOT SELECTED BY INDICES?].  However, one needs a fully characterized training set to use this method, in addition to the risk of over-fitting the model during the training process. [I DON'T UNDERSTAND WHAT YOU MEAN BY THIS SENTENCE]


%%%%%%%%%%%% ADAM STOPPED HERE %%%%%%%%%%%%


\section{Sample Characterization}\label{sec:results}

\subsection{ Absolute Magnitude-Spectral Type Relations }\label{sec:absmags}
Given the absence of reliable absolute F110W, F140W and, F160W absolute magnitude spectral type relations covering the spectral type range of M5--Y2 in literature, we created an absolute magnitude-spectral type relation to estimate distances of objects in our observed sample, created from $J$, $H$ relations by \cite{Pecaut2013}. We computed an offset/color between $J$ and $H$ magnitudes and AB F110W, F140W, and F160W magnitudes and used this offset to obtain new relations. We used a sample of 322 spectra the SpeX Prism library with measured parallaxes and measured the magnitude and a list of Y dwarfs from the Schneider sample [should change this to Y dwarfs with parallaxes] (Section~\ref{sec:trainset}) and measured their absolute magnitudes by convolving the flux-calibrated spectrum (the spectrum scaled to its absolute magnitude) with the filter profiles shown in \ref{fig:filterprofiles}.  The uncertainty in each convolution is computed by random sampling. The offset in convolutions between \textit{HST} filters and $J$ and $H$ filters is then added to the absolute magnitude-spectral type relations in $J$ and $H$ from \cite{Pecaut2013} to obtain an absolute F110W, F140W, and/or F160W magnitude. We then used a linear interpolation method to compute $J$ and $H$ magnitudes and a 6-degree polynomial fit to obtain the relations for F110W, F140W, and F160W. Error propagation for these steps is done by standard error propagation formula, i.e. $\sigma ^2 (\mathrm{F110W}-J)$= $\sigma ^2 (J) + \sigma ^2 (\mathrm{F110W})$ for instance. We report these relations and their intrinsic scatter in Table~\ref{tab:polynomials} and show them in Figure~\ref{fig:absmagrelations}.

\subsection{M dwarfs \& General Sample Statistics}
We found 121 objects with spectral types of M7--M9, these objects are defined by the \wat absorption features. We show the distance distribution of all the UCDs in the sample in Figure \ref{fig:candidedistances}, we find M and L dwarfs out to $\sim$ 3 kpc while T dwarfs are limited to within $\sim$ 500 pc. The observed galactic distribution of the UCDs is consistent with the galactic distribution and depths of the pointings in the survey, with a few more sources in the northern fields. Many of the M dwarf detections are robust i.e. the major \wat absorption feature between 1.3--1.45 \micron is distinguishable from noise in the spectrum. For WISP spectra, there are additional G102 data displaying other \wat and FeH and TiO features present in UCD atmospheres, confirming their spectral type classification. As an additional check, we visually inspected photometric images obtained using the coordinate positions from the photometric catalogs, and reference images of each pointing in respective filters to visually confirm point-sources and eliminate galaxies or other extended objects missed by the point-source cut and our selection methods. Nevertheless, we included some borderline objects in our sample that are possible UCDs based on their 1D G141 spectrum but with unidentifiable in the photometric image, given that source might have moved compared to the reference images used to derive positions of objects in the catalogs. That is the case for M7 GOODSN J1237+6210,  WISP J1224+6112, WISP J2005-4139, COSMOS J1000+0217, and WISP J1006-2953.

\subsection{ L \& T dwarfs}
\paragraph{Early L0-L5 Dwarfs}
We identified 26 early (L0--L5) L dwarfs in both surveys and 16 in WISP. There are four L0s all in WISP: WISP J1618+3340, WISP J0246-0104, WISP J1429+3224 WISP J1007+1004, and WISP J1715+0455. All four L0s are relatively good fits to the L0 spectral standards and have additional G102 data to confirm their spectral classification, in addition to being identifiable in their photometric cutouts. The WISP J1715+0455 G102 spectrum is noisier that its G141 spectrum, nevertheless, the fit to the spectral standard is not robust. These objects are located at distances of $\sim$ 800 pc, 1.3 kpc, 1.1 kpc, and 1.4 kpc, respectively. We found seven objects classified as L1s: UDS J0217-0509, WISP J0015-7955, WISP J1408+5657, GOODSS J0332-2742, WISP J1154+1941, WISP J0927+6027, and WISP J1150-203,3 all identifiable in their photometric images except for GOODSS J0332-2742 whose F160W image obtained from the catalog position points to a nearby extended object. Nevertheless, its 1D G141 shows molecular absorption features present in UCDs with a $J$-SNR of 5. As a feature of the WISP survey, G102 spectra are nosier than G141 spectra but the fit to the spectral standard remains robust. We estimated spectro-photometric distances of $\sim$ 700 pc, 1.1 kpc, 1 kpc, 1.2 kpc, 900 pc, 200 pc, and 260 pc, respectively for these sources. The next subtype is L2 and we identified two objects all in the GOODS Northern field: GOODSN J1236+6211 and GOODSN J1236+6209 are both identifiable by their G141 spectra and as point-sources in their photometric images, despite GOODSN J1236+6211 being in a crowded field. We estimated distances of $\sim$ 900 pc and 2.4 kpc for both, respectively. This makes GOODSN J1236+6211 with a $J$-SNR of 5, the most distant L/T dwarf in the sample. Next, we discovered three L3 dwarfs, all in the WISP fields: WISP J1544+4844, WISP J1154+1939 and WISP J1133+0328. The first two objects are borderline cases given their low $J$-SNR, but all three L3s show strong \wat absorption at $\sim$ 1.4 \micron\ confirming their spectral types. We estimated distances of 1 kpc, 1.3 kpc, and 600 pc, respectively. In terms of L4s, we found nine objects: WISP J0125-0001, COSMOS J1000+0219, GOODSS J0333-2751, WISP J1019+2743, WISP J1625+5721, GOODSN J1236+6209, WISP J1004+5258, GOODSN J1236+6214, GOODSS J0332-2749. Their spectral traces are all robust, except for WISP J1004+5258 which shows extra flux at 1.4 \micron\ from possible contamination with nearby objects. GOODSS J0333-2751 is the highest SNR object in this set and it shows a poor fit on both ends of spectrum that we attribute to contamination in the spectrum. These nine objects are located at distances of $\sim$ 1.1 kpc, 1.2 kpc, 500 pc, 750 pc, 500 pc, 1.9 kpc, 850 pc, 670 pc, and 1.2 kpc, respectively. Finally, GOODSN J1235+6211 is the lone L5 object in the sample located at $\sim$ 1 kpc with no particular interesting features and a low $J$-SNR of 6.

\paragraph{L/T transition objects}
There are two mid-L objects: WISP J1124+4202 and UDS J0217-0509 are both low-SNR objects but they have robust \wat features confirming their classification. They are located at $\sim$ 460 pc and 960 pc, respectively.

\paragraph{T dwarfs}
T dwarfs are characterized by stronger \wat absorption \meth features. In our sample, we identified 13 T dwarfs. WISP J1003+2854 is classified as a T0 at $\sim$1 kpc. The G141 spectrum displays deep \wat feature at 1.4 \micron, but the spectrum is a poor fit to spectral standards towards the edge of the detector. This object has a magnitude of F160W=23.1 placing it at $\sim$ 600 pc. COSMOS J1000+0217 is also classified as a T0 at a distance $\sim$900 pc. There are three T1 objects in the sample: WISP J1115+5257, WISP J0326-1643, and WISP J1150-2033 at distances of 950 pc, 770 pc and 620 pc, respectively. The first two objects have additional G102 spectra that further confirm their classification while WISP J1150-2033 does not appear in the photometric cutout, making it a borderline detection. For later types, the strength of the \meth absorption feature at $\sim$ 1.61 \micron is more pronounced. We found 2 T3 dwarfs: GOODSS J0332-2749 and WISP J0437-1106 at respective spectro-photometric distances of $\sim$ 450 and 820 pc respectively.  WISP J0437-1106 additional G102 data further confirms its spectral classification. In terms of mid-to late T dwarfs, we found 3 objects previously discovered by \cite{2012ApJ...752L..14M}. WISP0307-7243 is classified as T4 at $\sim$500 pc, WISP J1232-0033 is classified as T7 at $\sim$200 pc and  WISP1305-2538 is classified as T9 at $\sim$300 pc. Our classifications and distances agree with the previous classification. We found another T dwarf in AEGIS-03, AEGIS J1418+5242 is classified as T4, with a high SNR ($J$-SNR=21) and apparent magnitude of F140W $=22.7$ implying a distance of $\sim$500 pc. The spectrum is a good fit to the spectral standard and there is no visible contamination by nearby objects in the field or other spectral orders. Finally, GOODSS J0332-2741 is the latest T dwarf in the sample that has not been identified by other works. The 1D G141 spectrum of this spectrum fits to the T6 spectral standard, although the water feature at 1.2 \micron\ is stronger. There is also excess flux in the continuum at 1.6 \micron\ not present in the spectral standard pointing to a possible bad telluric correction in the standard and/or poor estimation of the contamination of this object. Nevertheless, with magnitudes of F140 $=22.1$ and F160W $=22.9$, we estimate a spectro-photometric distance $254\pm51$ pc for this object. 

\subsection{ Subdwarfs, Y dwarfs  \& Spectral Binaries}
We searched for subdwarfs and Y dwarfs by creating selection criteria for these subtypes. However, we did not find any obvious subdwarfs or binaries in the sample with the two methods. In general, the sample are relatively good fits to spectral standards with no peculiar excess fluxes that could not be attributed to the noise or the contamination in the spectrum. This is unsurprising given that estimates of the ratio of subdwarfs to dwarfs is 1/400 (ref) and the binary fraction of UCDs is very low \textless 10\%. 


\section{Probing Galactic Structure}\label{sec:simulations}

\subsection{ Point-Source Limiting Magnitudes}
\citet{Momcheva2016} reported the effective depths of all the pointings in 3D-HST, however, given various cuts that we made and variations in exposure times per pointing, we estimated a limiting magnitude for each individual pointing separately. We fitted a Gaussian kernel density estimator (KDE) to the distribution of magnitudes of point-sources with $J$-SNR $> 3$ in each filter, to obtain a probability distribution (PDF) of the magnitudes. The choice of using a KDE is advantageous for pointings with only a few point-sources, while using a simple histogram might be subject to visual biases depending on the width of the bins. The faint-limit is set at the maximum of the computed PDF while we set an automatic bright end at 16 based on the tail end of the overall distribution of magnitudes (Figure \ref{fig:maglimit}). These brightness and faintness limits are then used to estimate the effective volume of each pointing based on the absolute magnitude-spectral type relations defined in this work. 

\subsection{Monte-Carlo Simulation}
The observed number of stars in a given observational sample depends on the local luminosity function, the probed effective volume, and selection biases. For UCDs, it is also necessary to take into account brown dwarf evolution. We constructed a Monte-Carlo simulation to fully integrate these effects following methods based \cite{1999ApJ...521..613R} and \cite{2004ApJS..155..191B}, explained in this section.

\subsubsection{ Local Luminosity Function }
The local UCD luminosity function (LF=$\frac{dN}{dLog L}= \Phi_L$ in SpT$^{-1}$ pc$^{-3}$) of UCDs has been measured by several groups \citep{2003AJ....126.2449R,2007astro.ph..2034C,2010AJ....139.2679B,2008ApJ...676.1281M,Reyle2010a,2019ApJS..240...19K,2019arXiv190604166B}. \cite{2016AJ....151...92R} approached this problem using a parameterized LF matched to observations. Because our sample of UCDs probes large distances, we simulated a luminosity function from two fundamental stellar distributions: the mass function and the age distribution. 

We generated a sample of 10$^5$ objects from a power-law mass function parametrized by $\alpha$ for masses between 0.001 \Msun and 0.1 \Msun. \begin{equation}  P(M) = \frac{dN}{dM} \sim \biggl( \frac{M}{M_\sun}\biggl)^{-\alpha}\end{equation}. We adopt $\alpha=0.6$ based on results from \cite{2019ApJS..240...19K}, consistent with the mass function of UCDs in young clusters. Masses are drawn by inverting the cumulative distribution function (CDF) of the mass function as $M= CDF^{-1} (x)$ for $x \in$ [0, 1.].  To include the effects of binaries in our simulation, we assumed a power-law binary mass ratio given by 
\begin{equation}
    P(M1/M2)= \biggl(\frac{M1}{M2}\biggl)^{\gamma}
\end{equation} where $M1$ and $M2$ are the masses of the primary and assumed to follow the same underlying mass-function mentioned above. In addition, we assumed $\gamma$=4 based on the observed statistical distribution of UCD binaries \citep{2007ApJ...659..655B}. We assumed that all binaries in this simulation are unresolved, implying a binary fraction of $\epsilon$ = 0.2 [ref]. We finally assigned ages to these systems drawn from a uniform age distribution spanning 100 Myr--10 Gyr which reasonably matches the local stellar population. 

We determined effective temperatures, L$_{\text{bol}}$, $\log g$, and radii to each of the simulated objects, using a log-linear interpolation of solar-metallicity evolutionary model grids of three sets of models: \cite{2003A&A...402..701B} valid for stellar masses of 0.0005 to 0.1 \Msun\ and solar-metallicity hybrid models of \cite{2008ApJ...689.1327S} valid for stellar masses of 0.0002 to 0.85 \Msun. Evolutionary models parameters were assigned to each simulated object including objects in binary systems. The total mass of each binary is given by the sum of the masses of the primary and secondary, while the age and \teff of the binary system are the age and \teff of the most luminous object in the system.

To match these distributions to the measured luminosity function (LF), we scaled the temperature distribution of the generated UCD systems to the measured LF of \cite{2019ApJS..240...19K} of $\Phi$= 0.63 $\times$ 10 $^{-3}$pc$^{-3} K^{-1}$ for \teff values between 1650--1800 Kelvin and defined the scaling factor from the distribution of effective temperature $n$(\teff) as $N_0= 0.00063$ pc$^{-3}$ $\sum_{1650 K} ^ {1800 K}$ $n$(\teff). A comparison between the prediction from evolutionary models and the measured LF (Figure \ref{fig:lf}) shows a general agreement between models and the empirical LF of \cite{2019ApJS..240...19K} except for the parameter space where the models are invalid. As an additional check, we compared the luminosity functions of \cite{2019ApJ...883..205B} and \cite{2007astro.ph..2034C} to the predictions from models using a conversion between spectral type and absolute $J$ magnitude of \cite{2012ApJS..201...19D} by normalizing the distribution of magnitudes to $\Phi$ (J $\in$ [11.75, 12.25]) = 0.0015 mag $^{-1}$ pc$^{-3}$. Both sets of models show an agreement with the measurements of \cite{2019ApJ...883..205B}  within the domain of validity of the absolute-magnitude relations. Note that these models assume solar metallicity field predictions. As discussed further below, the ratio of the number of field objects to metal-poor halo and thick disk objects is too small to be detected in our sample of L \& T dwarfs ($\sim$40 objects).

\subsubsection{Spectral Type Classification \& Binaries}
Having computed a local luminosity function, we must then compute the observed volume to compare the number density of observations to simulations. The observed volumes of UCDs depends not only on the magnitude limit of the survey but the spectral type of the UCD. Hence, we computing an accurate spectral type classification is a crucial step. One complication is that this simulation does not generate a spectra, hence we must use polynomial relations that relate evolutionary model parameters (Lbol, \teff, age, mass) to spectral type, based on studies of local samples of UCDs.

For objects not binary systems, we simply converted  their effective temperatures to spectral types (M7--T8) using a linear interpolation of \teff and spectral types from \cite{Pecaut2013} and accounting for the scatter in this relation of $\Delta$ T$_{eff}$= 108 K, that is the temperature is chosen from a Gaussian centered around the interpolated value with a width of 108K. For objects in binary systems, we generated a table of composite spectral types and used this table to assign a classification. We assumed that UCDs in binary systems with less than [insert val] separation constitute [insert val] of the total number of binaries and could not be resolved by the WFC3 camera. To generate this table, we compiled a sample of possible bonafide singles in the SpeX Prism Library (SPL) excluding spectral binaries listed in \cite{2019arXiv190604166B}. We then picked a combination of primaries and secondaries where the spectral type of the primary is greater than that of the secondary. To generate a spectral binary, we scaled both the primary and the secondary spectra fluxes to their absolute magnitude, then created composite spectrum by adding the flux of the primary in the secondary. The composite spectrum's spectral type was then determined by comparing the spectrum to spectral standards using the method in section \ref{sec:classification}. 

\subsubsection{Effective Volumes}
The observed effective volume of each pointing depends on the scale height and the limiting magnitude of the survey. Having obtained the limiting magnitude of each pointing, we computed distance limits for a given spectral type and pointing $d_{\text{lim}}$ determined by 
\begin{equation} 
\log d_{\text{lim}} =\frac{1}{5}(m_{\text{lim}}-M(\text{SpT}))+1, 
\end{equation}
where m in the faint or the bright limit of the survey and $M$(SpT) is the absolute magnitude for that spectral type. We estimate the limiting distance in each available filter and obtain an effective limiting distance by averaging estimates in all filters using our absolute magnitude calculations described in Section~\ref{sec:absmags}. An accurate treatment of the limiting depth would account for the effect of dust extinction; however, 3D-HST and WISPS pointings are located at high enough galactic latitudes to avoid this issue in this study.

We assume a simple single-component axisymmetric exponential disk model parametrized by $\theta=(H, L)$ where $H$ and $L$ are the vertical scale height and radial scale length of the stellar number density given by
\begin{equation} 
\rho(\vec{r}) =\rho(R, z)= \rho _\sun \cdot \exp \biggl( {\frac{-|z-Z_\sun|}{H}} \biggl) \cdot \exp \biggl( {-\frac{R-R_\sun}{L}} \biggl),
\end{equation} 
where $R,z$ are cylindrical coordinates centered around the Galactic center, $L$ is assumed to be 2600 pc [SDSS] while $H$ was allowed to vary from values of 200 pc, 250 pc, 275 pc, 300 pc, 325 pc, 350 pc, and 1000 pc. $R_\sun$ and $Z_\sun$ are the sun's position from the galactic center, fixed at 8300 pc and 27 pc respectively. The vector $\vec{r}$ is the galacto-centric position vector and can be related to the star's distance from the Sun as 
\begin{equation} 
\vec{r} =  \biggl( R_\sun - d\cos(\beta) \cos(l), -d \cos(\beta) \sin(l), Z_\sun + d \sin (\beta)  \biggl), 
\end{equation} 
where $(\beta, l)$ are right-handed Sun-centric galactic coordinates for that star.

Given the Galactic structure model, we compute the effective volume of a pointing along a line of sight ($R, z$) as 
\begin{equation}
V_{eff}=\Delta \Omega \int _{d_{min}} ^{d_{max}} d|\vec{r}|  \cdot \frac{\rho(\vec{r})} {\rho _\sun}\cdot |\vec{r}|^2, 
\end{equation} 
where $\Delta \Omega$ is the solid angle of each pointing in this study fixed at the field of view of the WFC3 with $\Delta \Omega$ 3.47$\times$ 10$^{-7}$radian$^2$, $|\vec{r}|^2 = R^2 +z ^2$ is the distance along that line of sight, and $(d_{min}, d_{max})$ are the limiting depths. This is a one-dimensional integral along a line of sight ($\beta$, l).

After computing the effective volume of all the pointings in both surveys, we assigned a pointing to each of the simulated systems. We simply computed the normalized cumulative distribution of $V_{eff}$ for a given scale height, that is $CDF = 1/V_{tot}\sum V_{eff}  $, and assigned a given volume by drawing a number $CDF \in  [0, 1]$.  In other words, the likelihood of a pointing is given by its effective volume. \begin{equation} P(l, b) \sim  V_{eff}(l, b) \end{equation}. This distribution of volumes is shown in Figure \ref{fig:simvls}. 

\subsubsection{Selection Effects}
Because we applied several selection criteria to narrow down our sample for visual confirmation, it is possible we may have missed a few UCDs in the WISPS/3D-HST fields; particularly low SNR or peculiar objects due, in part, to uncertainties in spectral indices. Hence, the observed volumes must corrected by a factor proportional to our selection biases. To fully quantify these effects, we generated a distribution of low-resolution spectra uniformly sampling our SNR distribution across a wide range of SNRs and measured their recovery rate through this selection process by augmenting the SpeX sample to cover 3 orders of magnitude in SNR. To create this sample, we picked the top 20 highest SNR spectra with a median SNR between 50 and 200 and with spectral types between M7--T9 in the SpeX sample, obtaining a total of 298 spectra. We also added the WFC3 spectra of the Schneider set to the sample. We then degraded each spectrum by Gaussian iteration of $10^2$ creating a new sample of 21,800 spectra across the signal-to-noise range. Each new ``degraded" spectrum is created as 
\begin{equation} 
\{F(\lambda _i)\} \sim \text{Normal} (\langle F(\lambda_i)\rangle, \sigma^t(\lambda_i )), 
\end{equation} 
where $\sigma^t(\lambda_i )$ is the the target noise at the $i$-th wavelength $\lambda_i$, and $\langle F(\lambda_i )\rangle$ is the flux of the original spectrum at the $i$-th wavelength. We computed all relevant statistics for each of the degraded spectra, including $J$-SNR, spectral indices, $F$-test, and the two $\chi^2$s. We applied our selection processes to this sample of simulated spectra by measuring spectral indices and applying first $F$-test criterion where $F$-test \textless\ 0.4, box index-index selection criteria and the random forest classifier. 

With a perfect selection function, we would recover the entire ideal sample of spectra across the SNR range, however, we expect to lose some of the lowest $J$-SNR objects. Hence, we defined our selection probability in a given signal-to-noise range ($\Delta J$-SNR bin of 2.0 as $\mathcal{S}(J\text{-SNR, SpT})$)
\begin{equation}\label{equasl}
\mathcal{S}(J\text{-SNR}_i, \text{SpT})= \frac{N_{s, i} }{N_{tot, i}},
\end{equation} 
where $N_s$ is the number selected spectral type and SNR bin, and $N_{tot}$ is the total number of objects in that bin. $N_{s, i}$ is the number of objects in a bin ($i$) and $N_{tot, i}$ is the total number of spectra in that signal-to-noise ratio bin. For instance, if we simulated 100 M7 objects for $J$-SNR $\in$[10, 12] and recovered 70 objects classified as M7, then $\mathcal{S}$($J$-SNR $\in$ [10, 12], M7) $=0.7$. These selection probabilities for each selection method are showcased in Figure~\ref{fig:selectionf}. As expected the highest signal-to-noise objects/systems are selected across all spectral types, but we lose a few of them for low-SNR as indices become more uncertain. However, our selection methods turn out to be generous in selecting objects down to our SNR cut.

To apply this selection process to our Monte-Carlo simulation, we assign a distance to each of drawn from the Galactic structure model. The likelihood of distance ($d$) is 
\begin{equation}  
P(d) \sim \rho (d, \beta, l) \cdot d^2, 
\end{equation} 
where $\beta$ and $l$ are the direction a random pointing in 3D-HST or WISPS. We do not account for the directionality in this likelihood, they are drawn uniformly from the distribution of pointings in 3D-HST and WISPS for simplicity. For a given spectral type we assign a distance limited between $d_\mathrm{min}$/2 \textless\ $d$ \textless\ $5 \times d_\mathrm{max}$ to account for objects scattered in the observed volume. Samples for this part of the simulation were generated using \texttt{PyMC} for a number of samples $N=10^4$, sampling each scale height independently using a standard Metropolis-Hastings Algorithm. To make this step and the next steps pointing-independent, given that the limiting magnitudes are a function of pointing, we averaged the limiting magnitude and the limiting distances within each subtype.

With this distance distribution, and parameters of the surveys, we can estimate an observable signal-to-noise ratio in the J-band (J-SNR) as observed by the WFC3 instrument based on the observed UCD sample. We fit a second-degree polynomial to the observed magnitudes (F110W, F140W, F160W)  and SNR-J of our observed sample ans use our derived absolute-magnitude spectral type relations to estimate the apparent magnitude of each object in our simulated sample based on its randomly-assigned distance and spectral type. The apparent magnitudes (F110W, F140W), F160W) are then used to estimate a J-SNR ratio as observed with the WFC3 instrument. The final J-SNR is computed as the average of the three values. Objects outside the limiting magnitude of each pointing are not considered, hence they are assigned a selection probability ($\mathcal{S}$) of zero.

Given a $J$-SNR and SpT, we can now assign a selection probability $\mathcal{S}$ to each object. We computed the expected number of objects per spectral type in given pointing by a simple product of selection probabilities of objects in that pointing and the effective volume scaled by the normalization factor
\begin{equation}
N_{sim}(\text{SpT})= N_0 \cdot V_{eff} (\text{SpT}) \cdot \sum _i \mathcal{S}(J\text{-SNR}_i, \text{SpT}). \end{equation} 
We compared these numbers to the observed numbers of UCDs for each age distribution in Figure~\ref{fig:simulationnbrs}.


\subsection{Results}
The resulting \teff distribution (Figure~\ref{fig:lf}) from interpolating mass and ages onto both evolutionary models grids is consistent with expectations given atmospheric cooling effects \citep{2004ApJS..155..191B}. As UCDs age, they quickly pile up on at the lower end of the spectral type distribution (i.e. cooler temperatures), though \cite{2008ApJ...689.1327S} models incorporate clouds and predict a bump at \teff $\sim$ 1200 K. Figure~\ref{fig:simulationnbrs} shows the effect of scale height on the probed effective volume. The predicted number density (Figure~\ref{fig:simulationnbrs}) matches scale height assumptions of $\sim$ 200--300 pc for late M and early L dwarfs, and $\sim$ 200--400 pc for early T dwarfs, which are consistent with compiled literature values. For L/T transition objects, we predict a slightly lower scale height and a slightly larger scale height for early T dwarfs compared to other typical members of their respective spectral classes. This mismatch between observations and predictions is perhaps due to a failure to truly resole the nature of L/T transition phase using our interpolation methods. Moreover, the L/T transition region is sensitive to unresolved binaries \citep{2014ApJ...794..143B} and \cite{2007ApJ...659..655B} shows this effects causes an increase in the number of density of T0--T5 up to 10--15\%. Metallicity effects affect the number of subdwarfs we expected in this sample. UCDs in the thick disk and the halo have similar kinematic ages with stellar populations in these parts of the Galaxy; and UCDs at different metallicities follow different evolutionary tracks. L subdwarfs in the local neighborhood are therefore rare, and this study does not significantly probe large volumes in the thick disk and halo. \cite{Lodieu2017} found 0.04 deg$^{-2}$ L subdwarfs in the UKIDSS/SDSS fields; in fact, we expect the number of subdwarfs to be $\sim$ 400 times lower than the expected number of dwarfs in the sample. Nonetheless, the scale height of L5--T0 and late T dwarfs remains unconstrained by this sample given their small number. 

\section{Summary\label{sec:summary}}
%summarize the selection process
The WISPS \& 3D-HST surveys provide NIR G141 (1.1--1.14 \micron) spectroscopic data and broadband F140W, F110W, and F160W photometry for thousands of galaxies and point-sources observed in parallel mode with other on-going HST surveys. We made a point-source cut using in the surveys and obtained 271915 point-sources. Using NIR spectral indices that sample the prominent \wat and \meth absorption features in UCD atmospheres, we created selection criteria based on a calibration sample of templates. We have presented two methods for selecting UCDs in deep HST surveys potentially applicable future infrared parallel surveys. Both methods rely on spectral indices defined to trace \wat and \meth features prominent in the NIR band of UCDs. The box selection method is efficient (completeness \textgreater\ 90\%) but with relatively high contamination rates that could be significantly reduced by eliminating the lowest SNR sources. This method is not effective for selecting very low SNR sources due to large scatter in indices and early M dwarfs as the absorption features in these wavelength ranges are shallow. However, these spectral indices are designed to selected T dwarfs with high accuracy (completeness \textgreater 90\%, contamination \textless 1\%). The overall contamination/false positive rate for this method for spectral types of L0--L5 is $\sim$ 87\% . A second method uses a random forest classifier to distinguish UCDs from other extragalactic contaminants or artifacts with an accuracy score of 99.5\% in cross-validation. The false positive rate of this method  for spectral types of L0--L5 is $\sim$62\%. Both methods rely on a training set of known UCD samples and can be combined. With these two methods, we have used these data to obtain 166 spectra of M7--T9 UCDs up to distances $\sim$ 4 kpc. 

%summarize the simulation results
We estimated the expected number of UCDs given a galactic structure model with scale height ($h$) as a free-parameter. Using a point-source limiting magnitude, we measured the effective volumes of the survey for various values of the scale height. To address intrinsic biases in our selection method, we use a Monte-Carlo simulation to reproduce a distribution of spectral type based on a set of fundamental distribution: mass function, age distribution and conversion/polynomial relation from UCD evolutionary models and our sample. We use the galactic structure model to draw a distribution of distances. With these distributions, we create a selection probability function based on sample of ``degraded" templates. The final steps involve summing over selection probabilities. The predicted number of UCDs is consistent with a scale height of 325 pc $\leq h \leq$ 350 pc.  

%Implications for JSWT (quote ryan 2016)
Future space missions such as \textit{JSWT} and \textit{Euclid} will be contaminated by UCDs. \cite{2016AJ....151...92R} predicted that the number density of UCDs (M8--T8) in JSWT fields peaks around $J\sim24$ mag with a total surface density of $\Sigma$ $\sim$ 0.3 arcmin$^{-2}$.  With the \textit{Large-Scale Synopitc Telescope} (LSST), and the \textit{Wide-Field Infrared Survey Telescope} (\textit{WFIRST}), we expect an increase in both sample size and spectral type accuracy, expanding the parameter space necessary to put significant constraints on the star formation history of the Milky Way and the mass function of UCDs \citep{LSSTScienceCollaboration2009,Spergel2015}.

%Implications for Euclid deep fields


\acknowledgements
%Acknowledgements
%Wisps funding
%3D-HST 
%LSSTC-DSFP
%Software 
This work is based on observations taken by the 3D-HST treasury program (GO 12177 and 12328) with the NASA/ESA HST, which is operated by the Association of universities for Research in Astronomy, Inc. under NASA contract NAS5-26555.

CA thanks the LSSTC Data Science Fellowship Program, which is funded by LSSTC, NSF Cybertraining Grant \#1829740, the Brinson Foundation, and the Moore Foundation; his participation in the program has benefited this work.

\software{Astropy \citep{Collaboration2013}, 
		Matplotlib \citep{4160265},
		 SPLAT \citep{Burgasser2014}, 
		 Scipy \citep{2019arXiv190710121V}, 
		 Pandas, 
		 Seaborn \citep{michael_waskom_2014_12710}, 
		 PyMC3 \citep{10.7717/peerj-cs.55} }

\clearpage

\bibliography{bibl.bib}
\end{document}


